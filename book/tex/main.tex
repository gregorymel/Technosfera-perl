\documentclass{report}
\usepackage{expl3,xargs,comment,luacode}
\usepackage[table]{xcolor}
\usepackage{polyglossia,fancyvrb}
\usepackage{enumitem,graphicx,float} %subcaption
\usepackage{amsmath,amssymb,mathtools}
\usepackage{geometry}
\geometry{margin=2cm}


\usepackage{titlesec}
\usepackage{titleps}
\definecolor{gray75}{gray}{0.75}
\newcommand{\hsp}{\hspace{20pt}}


\setdefaultlanguage{russian}
\setotherlanguage{english}
\setmainfont{CMU Serif}
\setsansfont{CMU Sans Serif}
\setmonofont{CMU Typewriter Text}


\usepackage[unicode]{hyperref}
\definecolor{linkcolor}{HTML}{000000} 
\definecolor{urlcolor}{HTML}{333377}
\hypersetup{pdfstartview=FitH, linkcolor=linkcolor, urlcolor=urlcolor, colorlinks=true}

\usepackage{tikz}
\usetikzlibrary{fadings,scopes,chains,decorations,decorations.pathreplacing,angles,calc,quotes,positioning}
\usepackage{pgfplots,unicode-math,wasysym}

\usepackage{shellesc}

\usepackage[outputdir=build]{minted}
\setminted{breaklines=true}

\addto\captionsrussian{%
  \renewcommand{\chaptername}{Лекция}
}


\expandafter\def\csname ver@transparent.sty\endcsname{}
\expandafter\def\csname ver@subfig.sty\endcsname{}
\usepackage{svg}


%\makeatletter
%\expandafter\def\csname PYGdefault@tok@err\endcsname{\def\PYGdefault@bc##1{{\strut ##1}}}
%\makeatother
\AtBeginEnvironment{minted}{\renewcommand{\fcolorbox}[4][]{#4}}

\makeatletter
\input{common/tikz/common_debug.tikz}
\input{common/tikz/tikz_svhead.tikz}
\input{common/tikz/tikz_stackblock.tikz}
\makeatother
\debugfalse


\tikzset{
grammar style/.style={
  text depth=.25ex,
  line width=2pt,
  draw=black,
  rounded corners=3mm,
% Настройка типов нодов:
  common/.style  = { minimum size=6mm,  draw=black!50, font=\large\rm\bf,
                      top color=white,  bottom color=green!20},
  value/.style          = { common, draw,
                            minimum width=25mm,
                            rounded corners=0mm      },
  exotic value/.style   = { value,  font=\large\rm,
                            rounded corners=3mm      },
  operator/.style       = { common, circle           },
% Макрос, который устанавливает референтные точки автоматически
  setSyntaxDiagramPoints/.code = {
    \path (-6,0) coordinate (LEnd)      ( 6,0) coordinate (REnd)
          (-5,0) coordinate (LBracket)  ( 5,0) coordinate (RBracket)
        (-3.8,0) coordinate (L)       ( 3.8,0) coordinate (R)
        (-4.1,0) coordinate (LP)      (4.1 ,0) coordinate (RP)
        (0, 0.8) coordinate (UP)      (0, -1)  coordinate (DW);
  }
}}


\input{tikz/tikz-svschemes.tikz}

%\includeonly{lectures/L7/L7,tex/L10}
\newcommand\perldoc[1]{\href{http://perldoc.perl.org/#1.html}{#1}}

\begin{document}
\tableofcontents
%\include{lectures/L1/L1}
%\include{lectures/L2/L2}
%\include{lectures/L2/L2-bonus}
%\setcounter{chapter}{2}
\chapter{Модульность и повторное использование}
Данная лекция посвящена модульности в языке \verb|perl|. Знания базового синтаксиса языка программирования недостаточно, чтобы писать сложные законченные программные продукты, поскольку любой полноценный проект состоит из множества модулей. Построение сложной иерархии проекта с технической и логической точек зрения будет темой данной лекции.

\section{Команды типа <<include>>} %1 (1:21)

\subsection{eval}
В языке C команда \verb|include| позволяет подключить другой файл с кодом с помощью <<механической>> подстановки его содержимого. Эта команда позволяет разбивать сложные проекты на несколько файлов. Похожие команды есть и в любом другом языке.

Самый простой способ исполнить некоторый код в языке \verb|perl| --- использовать функцию \verb|eval|. Если передать этой функции строку с кодом, этот код будет исполнен.
\begin{minted}{perl}
my $u;

eval '
  $u = 5;
  my $y = 10;
  sub m_3 {
    my ($x) = @_;
    return $x * 3;
  }
';

$u; # 5
$y; # Undefined
m_3(2); # 6
\end{minted}
Важной особенностью является то, что \verb|eval| создает свою область видимости, а, следовательно, локальные переменные, объявленные с помощью \verb|my|, будут ограничены функцией \verb|eval|. Это можно заметить в приведенном выше примере: переменная $\$y$ была объявлена с помощью my и существует только внутри \verb|eval|, а переменная $\$u$, объявленная вне \verb|eval|, внутри \verb|eval| изменяет свое значение.

Этот способ прост, но требует выполнения множества дополнительных действий вручную. К таким действиям относятся, например, чтение файлов и обработка ошибок.

\subsection{do} %3 (3:05)
\verb|do| --- более совершенная версия \verb|eval|. Не следует путать этот \verb|do| с \verb|do|, который используется для создания циклов. Функция \verb|do| принимает имя файла, сама его считывает и исполняет с помощью \verb|eval|.
\begin{minted}{perl}
do 'sqr.pl';
\end{minted}
\begin{minted}{perl}
# sqr.pl
$u = 5;
my $y = 10;
sub m_3 {
  my ($x) = @_;
  return $x * 3;
}
\end{minted}
\begin{minted}{perl}
$u; # 5
$y; # Undefined
m_3(2); # 6
\end{minted}
Функция \verb|do|, как и \verb|eval| создает свою область видимости. Но и \verb|do| практически не используется в сложных проектах, так как существует более высокоуровневая функция \verb|require|.

\subsection{require} %4 (3:58)
Функция \verb|require| --- более совершенная высокоуровневая форма \verb|do|. Ей так же нужно передать имя файла, после чего произойдет импорт и последующее исполнение кода из файла, однако с некоторыми особенностями.

Во-первых, \verb|require| поддерживает синтаксис с использованием двойного двоеточия, что позволяет абстрагироваться от реальных имен файлов и давать модулям названия. Например, в данном случае:
\begin{minted}{perl}
require 'sqr.pl';
require Local::Sqr; # Local/Sqr.pm
\end{minted}
будет загружен файл не с разрешением \verb|.pl|, а файл с разрешением \verb|.pm| (perl module):
\begin{minted}{perl}
# Local/Sqr.pm
$u = 5;
my $y = 10;
sub m_3 {
  my ($x) = @_;
  return $x * 3;
}

1; # return value!
\end{minted}
Во-вторых, функция \verb|require| проверяет, что код модуля возвращает истинное значение. Поэтому основная масса модулей заканчивается строчкой <<1;>>:
\begin{minted}{perl}
$u; # 5
$y; # Undefined
m_3(2); # 6
\end{minted}
Это гарантирует, что последнее выражение в модуле истина и \verb|require| сочтет такой модуль успешно загруженным. В очень малом числе случаев модулю действительно нужно сообщить, успешно ли он загружен --- в этом случае используются специальные проверки.

Синтаксис с двойными двоеточиями позволяет указать путь до модуля:
\begin{minted}{perl}
require Module; # Module.pm
require Module::My; # Module/My.pm
\end{minted}
Данный код сработает вне зависимости от операционной системы и используемого в ней разделителя каталогов.

Поиск модулей \verb|require| выполняет в каталогах, содержащихся в массиве $@INC$ (на самом деле --- функция \verb|do| делает тоже самое):
\begin{minted}{bash}
perl -e 'print join "\n", @INC'
/etc/perl
/usr/local/lib/perl/5.14.2
/usr/local/share/perl/5.14.2
/usr/lib/perl5
/usr/share/perl5
/usr/lib/perl/5.14
/usr/share/perl/5.14
/usr/local/lib/site_perl
\end{minted}
Добавить каталог (чтобы \verb|require| искал модуль в нем) в этот массив можно (кроме непосредственной работы с ним) следующими способами:
\begin{enumerate}
	\item Добавив каталог в переменную окружения PERL5LIB:
  \begin{minted}{perl}
  $ PERL5LIB=/tmp/lib perl ...
  \end{minted}
	\item Используя ключ I интерпретатора:
  \begin{minted}{bash}
  $ perl -I /tmp/lib ...
  \end{minted}
\end{enumerate}
В данной главе были рассмотрены основные способы подключения модулей. Существует, однако, ещё один способ, о котором будет сказано позднее.

% ------------------------------------------------------
\section{Блоки фаз} %7:56
В \verb|perl| сушествует возможность указать блок \verb|BEGIN|, который будет исполнен в начале программы вне зависимости от реального расположения внутри программы:
\begin{minted}{perl}
BEGIN {
  require Some::Module;
}

sub test1 {
  return 'test1';

* sub test2 {
*   return 'test2';
*
*   BEGIN {...}
* }
}
\end{minted}
Важной особенностью \verb|perl| является то, что функции объявляются еще до исполнения программы. В данном примере будет сначала выполнен первый блок \verb|BEGIN|, потом объявлены функции $test1$ и $test2$, выполнен, вложенный в функцию $test2$, блок \verb|BEGIN| и только после этого начнется исполнение программы.

Парный блоку \verb|BEGIN|, блок \verb|END| исполняется, наоборот, когда программа завершилась:
\begin{minted}{perl}
open(my $fh, '>', $file);

while (1) {
  # ...
}

END {
  close($fh);
  unlink($file);
}
\end{minted}
Он исполняется последним вне зависимости от расположения в исходном коде программы. Чаще всего блок \verb|END| используется для очистки ресурсов. В данном примере в блоке \verb|END| реализован процесс завершения работы с файлом.

Также в \verb|perl| существуют блоки:
\begin{itemize}
	\item CHECK\{\}
	\item UNITCHECK\{\}
	\item INIT\{\}
\end{itemize}
Такое большое количество разнообразных блоков фаз связано с работой интерпретатора. Использование нужного блока позволяет исполнить требуемый код в нужный момент работы интерпретатора. Эти особенности далее обсуждаться не будут. В реальном коде данные блоки встречаются крайне редко.

Внутри всех блоков присутствует переменная
\[ \$\{ \textasciicircum GLOBAL \_ PHASE \}, \]
в которой хранится название текущей фазы (\verb|INIT|, \verb|UNITCHECK| и т.п.).

\section{Команды типа <<include>> (продолжение)} %10 (11:20)
Использование ключевого слова \verb|use| --- основной способ подключения модулей в \verb|perl|, которы представляет собой выполнение \verb|require| внутри блока \verb|BEGIN|. Модули подключаются в заданном порядке.
\begin{minted}{perl}
use My_module;     # My_module.pm
use Data::Dumper;  # Data/Dumper.pm
BEGIN { push(@INC, '/tmp/lib'); }
use Local::Module; # Local/Module.pm
\end{minted}
В данном случае сначала будут подключены два модуля, затем выполнен блок \verb|BEGIN|, а после --- подключен третий модуль.

Как и \verb|require|, \verb|use| умеет понимать литералы с двойными двоеточиями.

Выполнить \verb|use| можно используя ключ интерпретатора $-M$.

\section{Пространства имен} %11 (12:19)

\begin{minted}{perl}
require Some::Module;
function(); # ?

require Another::Module;
another_function(); # ??

require Another::Module2;
another_function(); # again!?
\end{minted}

В \verb|perl| пространства имён (англ. \verb|namespace|) называются пакетами (англ. \verb|package|). С помощью пакетов можно создать отдельную область видимости для функций и переменных так, чтобы они не были доступны извне по свои коротким именам, но доступны по $full$ $qualified$ $name$.

Ключевое слово \verb|package| используется для объявления пакета и все объявленные функции и переменные до конца области видимости будут входить в этот пакет.
Для имен пакетов используется такой же синтаксис с двумя двоеточиями, который встречался ранее. Это сделано не случайно --- существует соглашение внутри каждого модуля определять пакет с точно таким же именем. Это позволяет удобно организовать код программы.
\begin{minted}{perl}
require Some::Module;
Some::Module::function();

require Another::Module;
Another::Module::another_function();

require Another::Module2;
Another::Module2::another_function(); # np!
\end{minted}

Например, в следующем примере после подключения модуля Local::Multiplier
\begin{minted}{perl}
use Local::Multiplier;

print Local::Multiplier::m3(8); # 24
\end{minted}
имеется возможность использовать функции, объявленные в одноимённом пакете:
\begin{minted}{perl}
package Local::Multiplier;

sub m2 {
  my ($x) = @_;
  return $x * 2;
}

sub m3 {
  my ($x) = @_;
  return $x * 3;
}
\end{minted}
Имена функций отделяются от имени пакета также двойным двоеточием.

Ключевое слово \verb|package| не обязательно указывать в начале файла. Оно может быть использовано в любом месте и помещает переменные и функции в пакет до конца области видимости. Сразу после этого пакет становится доступен. Например:
\begin{minted}{perl}
{
  package Multiplier;
  sub m_4 { return shift() * 4 }
}

print Multiplier::m_4(8); # 32
\end{minted}
 Имя пакета можно получить используя ключевое слово \verb|__PACKAGE__|:
\begin{minted}{perl}
package Some::Module::Lala;

print __PACKAGE__; # Some::Module::Lala
\end{minted}

\section{Переменные пакета}
Переменные пакета объявляются с помощью ключевого слова \verb|our| (а не \verb|my|) и внутри пакета доступны по короткому имени. К переменным пакета можно обращаться всегда и по длинному имени, но это часто не удобно (например, при переименовании пакета пришлось бы переименовывать все его переменные).
\begin{minted}{perl}
{
  package Some;
  my $x = 1;
  our $y = 2; # $Some::y;

  our @array = qw(foo bar baz);
}

print $Some::x; # ''
print $Some::y; # '2'

print join(' ', @Some::array); # 'foo bar baz'
\end{minted}
Ключевое слово \verb|our| может также использоваться по отношению к массивам и хешам. В этом случае массив будет доступен по короткому имени внутри пакета, как будто это локальная переменная. Объявленные таким образом переменные будут все ещё доступны снаружи пакета по полным именам.

%\subsection{my,state} %16 (18:28)
Следует напомнить, что \verb|my| задает переменную в локальной области видимости:
\begin{minted}{perl}
my $x = 4;
{
  my $x = 5;
  print $x; # 5
}
print $x; # 4
\end{minted}
Если локальная переменная имеет то же имя, что и некоторая переменная из более широкой области видимости, то последняя в локальной области видимости оказывается недоступной.

С помощью конструкции
\begin{minted}{perl}
use feature 'state';
\end{minted}
можно включить возможность определять переменные с помощью ключевого слова \verb|state|. Основное отличие от \verb|my| состоит в том, что присваивание переменной значения происходит только раз за все время исполнения программы:
\begin{minted}{perl}
sub test {
  state $x = 42;
  return $x++;
}

printf(
  '%d %d %d %d %d',
  test(), test(), test(), test(), test()
); # 42 43 44 45 46
\end{minted}
В данном примере в функции \verb|state| переменной $\$x$ значение 42 присваивается только при первом ее вызове.

\section{Глобальная область видимости} %17 (20:04)
Глобальная область видимости является пакетом с именем \verb|main| (или имя которого --- пустая строка). Например, принудительно использовать переменную из глобальной области видимости можно указав в качестве имени пакета \verb|main|:
\begin{minted}{perl}
our $size = 42;

sub print_size {
  print $main::size;
}

package Some;
main::print_size(); # 42
\end{minted}
В некоторых пакетах глобальные переменные определяются в начале файла, а потом используются в пакете именно таким образом.

\section{Передача параметов} %18 (20:52)
Функция \verb|use|, кроме того, что исполняется в \verb|BEGIN|, поддерживает передачу параметров. Для этого после имени модуля указывается список параметров, которые будут переданы загрузчику модуля:
\begin{minted}{perl}
use Local::Module ('param1', 'param2');
use Another::Module qw(param1 param2);
\end{minted}
Другими словами, при исполнении \verb|use| не просто выполняется \verb|require| внутри блока \verb|BEGIN|, а также вызвается метод \verb|import| одноименного пакета (если он есть), которому собственно и передаются параметры:
\begin{minted}{perl}
BEGIN {
  require Module;
  Module->import(LIST);
  # ~ Module::import('Module', LIST);
}
\end{minted}
Это еще одна причина использовать одинаковые имена пакетов и модулей: функция \verb|use| будет искать метод \verb|import| в пакете, имя которого совпадает с именем загружаемого модуля.

Если метод \verb|import| вызывать не требуется вообще, после имени модуля нужно указать пустые скобки. Следует отметить, что указание пустых скобок и отсутствие скобок --- не одно и тоже. Если скобки отсутствуют, метод \verb|import| вызывается без параметров. Пустые скобки --- прямое указание на запрет вызова \verb|import|.

\section{Экспорт} %19 (23:17)
Название метода \verb|import| исходит из того, что в начале он использовался, чтобы выборочно импортировать некоторые функции из пакета. Например, в модуле \verb|File::Path|, одном из основных модулей \verb|perl|, есть функции $make\_path$ (создать множество вложенных каталогов) и $remove\_tree$ (удалить множество каталогов). Если включить \verb|use File::Path| и в качестве параметров передать \verb|qw(make_path remove_tree)|, то в текущем пакете (в данном случае \verb|main|) появятся функции $make\_path$ и $remove\_tree$:
\begin{minted}{perl}
package My::Package;

use File::Path qw(make_path remove_tree);

# File::Path::make_path
make_path('foo/bar/baz', '/zug/zwang');
File::Path::make_path('...');
My::Package::make_path('...');

# File::Path::remove_tree
remove_tree('foo/bar/baz', '/zug/zwang');
File::Path::remove_tree('...');
My::Package::remove_tree('...');
\end{minted}
После этого, так как метод \verb|import| был написан соответствующим образом, можно будет обращаться к данным функциям не только по полным именам, но и по коротким. Это самый простой и распространённый способ экспорта функций, поскольку позволяет контролировать информацию о полученных функциях и избегать конфликта имен с функциями из других модулей.

Поскольку все модули используют такой механизм, он реализован в отдельном модуле \verb|Exporter|:
\begin{minted}{perl}
package Local::Multiplier;

use Exporter 'import';
our @EXPORT = qw(m2 m3 m4 m5 m6);

sub m2 { shift() ** 2 }
sub m3 { shift() ** 3 }
sub m4 { shift() ** 4 }
sub m5 { shift() ** 5 }
sub m6 { shift() ** 6 }
\end{minted}
В пакете объявляется массив функций, которые требуется экспортировать, вызывается через \verb|use Exporter|, и, как только где-то будет вызван данный модуль, все функции, указанные в массиве, будут экспортированы.
\begin{minted}{perl}
use Local::Multiplier;

print m3(5); # 125
print Local::Multiplier::m3(5); # 125
\end{minted}

Чтобы иметь возможность выбирать, какие функции будут экспортированы, следует использовать массив $@EXPORT\_OK$:
\begin{minted}{perl}
package Local::Multiplier;

use Exporter 'import';
our @EXPORT_OK = qw(m2 m3 m4 m5 m6);

sub m2 { shift() ** 2 }
sub m3 { shift() ** 3 }
sub m4 { shift() ** 4 }
sub m5 { shift() ** 5 }
sub m6 { shift() ** 6 }
\end{minted}
В нем указываются функции, которые могут быть экспортированы. Какие из них будут экспортированы, выбираются по короткому имени при вызове данного модуля.
\begin{minted}{perl}
use Local::Multiplier qw(m3);

print m3(5); # 125
print Local::Multiplier::m4(5); # 625
\end{minted}

Также модуль \verb|Exporter| поддерживает экспорт групп функций. Для этого необходимо задать хеш $\%EXPORT\_TAGS$:
\begin{minted}{perl}
our %EXPORT_TAGS = (
  odd  => [qw(m3 m5)],
  even => [qw(m2 m4 m6)],
  all  => [qw(m2 m3 m4 m5 m6)],
);
\end{minted}
Чтобы экспортировать из модуля определенную группу функций используется синтаксис с двоеточием:
\begin{minted}{perl}
use Local::Multiplier qw(:odd);

print m3(5);
\end{minted}
К данному моменту уже перечислены основные механизмы подключения модулей.

\section{Загрузка определенной версии пакета}
Perl поддерживает загрузку пакета строго определенной версии. Для этого требуемую версию необходимо указать сразу после имени пакета:
\begin{minted}{perl}
use File::Path 2.00 qw(make_path);
\end{minted}
Версия пакета указывается внутри пакета в специальной переменной $\$VERSION$ следующим образом:
\begin{minted}{perl}
package Local::Module;

our $VERSION = 1.4;
\end{minted}
Если в это переменной будет значение меньше запрашиваемого, будет выведено сообщение об ошибке:
\begin{minted}{perl}
use Local::Module 1.5;
\end{minted}
\begin{minted}{bash}
$ perl -e 'use Data::Dumper 500'
Data::Dumper version 500 required--
this is only version 2.130_02 at -e line 1.
BEGIN failed--compilation aborted at -e line 1.
\end{minted}

На самом деле при проверке версии пакета вызвается метод \verb|VERSION| этого пакета: %24 29:19
\begin{minted}{perl}
use Local::Module 500;
# Local::Module->VERSION(500);
# ~ Local::Module::VERSION('Local::Module', 500);
\end{minted}
Внутри модуля эту функцию можно определить произвольным образом и тем самым задать то, как происходит проверка версии.
\begin{minted}{perl}
package Local::Module;

sub VERSION {
  my ($package, $version) = @_;

  # ...
}
\end{minted}
Такая функциональность требуется крайне редко.

Уже достаточно давно в \verb|perl| присутствует синтаксис для так называемых \verb|version strings|:
\begin{minted}{perl}
use Local::Module v5.11.133;
\end{minted}
Синтаксис состоит в том, что после имени модуля ставится символ <<v>> и сразу за ним несколько чисел, разделенных точками.
\begin{minted}{perl}
v102.111.111; # 'foo'
102.111.111;  # 'foo'
v1.5;
\end{minted}

Такая запись превращается в последовательность символов. Количество символов в последовательности равняется количеству чисел, а каждый символ в строке таков, что его код равен соответствующему числу. Такое преобразование версии в строку позволяет производить корректное лексикографическое сравнение двух версий. При лексикографическом сравнении сначала сравниваются первые символы двух строк, потом вторые и так далее. Так и при сравнении двух номеров версий сначала будут сравниваться старшие номера версий, и, если они совпадают, следующий за старшим и так далее.

Об этом не стоило бы и говорить, если бы не одно с этим связанное недоразумение. Дело в том, что если переменные или ключи хешей похожи на v-string, интерпретатор может принять эту запись за v-string и сделать вышенаписанное преобразование. Это следует иметь в виду и брать v и числа в кавычки, поскольку иначе запись будет неправильно воспринята интерпретатором.

\section{Указание версии интерпретатора}
При использовании команды \verb|use| можно указать номер версии, но не указывать название пакета. В таком случае будет указана требуемая версия интерпретатора \verb|perl|, а также станут доступны все возможности, которые появились в этой версии:
\begin{minted}{perl}
use 5.12.1;
use 5.012_001;

$^V # v5.12.1
$]  # 5.012001
\end{minted}
Версия интерпретатора хранится в переменной $\$\^V$ (в новом формате) и переменной $\$]$ (в старом формате):
\begin{minted}{perl}
use Module v1.1.1;
use 5.10;
\end{minted}
Именно с этой версией и будет сравниваться указанная после \verb|use| версия.

\section{Pragmatic modules} %27 31:35
С помощью \verb|use| можно загружать так называемые \verb|pragmatic modules|. От обычных модулей они отличаются тем, что (условно) влияют на ход интерпретации программы и их имена традиционно начинаются с маленькой буквы. Однако, строго говоря, какой-то конкретной границы между такими и обычными модулями нет.

Чаще всего используются два pragmatic модуля: \verb|strict| и \verb|warnings|.
\begin{minted}{perl}
use strict;
use warnings;
\end{minted}
Модуль \verb|strict| позволяет включать дополнительные ограничения, а модуль \verb|warnings| включает предупреждения.

\subsection{Модуль strict} %28 33:24
Если параметры не указаны, модуль \verb|strict| включает все три доступных типа ограничений. Фактически, \verb|use strict| --- это \verb|use strict 'refs'|, \verb|use strict 'vars'|, \verb|use strict 'subs'| вместе взятые. В результате этого некоторые опасные возможности языка становятся недоступными для программиста. Код же, который не использует эти возможности, становится более чистым и надежным.

Использование \verb|use strict 'refs'| позволяет избежать следующей нежелательной ситуации. Следует напомнить, что, если разыменовать указатель на переменную, то получается значение этой переменной:
\begin{minted}{perl}
use strict 'refs';

$ref = \$foo;
print $$ref;  # ok
$ref = "foo";
print $$ref;  # runtime error; normally ok
\end{minted}
Если не указано \verb|use strict 'refs'|, то результатом разыменования переменной-строки будет значение переменной, имя которой есть данная строка (строка может, например, быть считана из стандартного ввода или стороннего файла). Это немного странно и \verb|use strict 'refs'| запрещает такое поведение. Однако иногда такое поведение необходимо, поэтому этот режим можно отключить (об этом будет сказано позднее).

%\subsection{use strict 'vars'} %29 (34:43)
С помощью \verb|use strict 'vars'| можно потребовать явной инициализации переменной с помощью ключевых слов \verb|my| или \verb|our|. Если \verb|use strict 'vars'| не использовать, то обращение (без указания \verb|my| или \verb|our|) в начале файла к переменной $\$x$, фактически, будет обращением к переменной $\$main::x$ (к глобальной переменной), а не к локальной переменной, как это, возможно, задумывалось.

% WARNING: Может поменять местами куски?
С помощью \verb|use strict 'subs'| отключается автоматические перевод bareword'ов (слово без кавычек) в строки:
\begin{minted}{perl}
use strict 'vars';
$Module::a;
my  $x = 4;
our $y = 5;
\end{minted}
Например, если \verb|use strict 'subs'| не используется и функция $test$ не определена:
\begin{minted}{perl}
use strict 'subs';
print Dumper [test]; # 'test'
\end{minted}
Если же до этого определить функцию $test$, поведение совершенно меняется:
\begin{minted}{perl}
sub test {
  return 'str';
}
print Dumper [test]; # 'str'
\end{minted}
Подход, в котором то, как будет интерпретироваться bareword, зависит от того, какие функции существуют на момент исполнения кода, является неприемлемым (кроме, может быть, в случае однострочников).

\subsection{Модуль warnings} %37:10
Модуль \verb|warnings| включает отображение предупреждений только в данной области видимости (в отличие от ключа <<w>> интерпретатора):
\begin{minted}{perl}
use warings;
use warnings 'deprecated';
\end{minted}
Использовать модуль \verb|warnings| более правильно, так как он не включает предупреждения в модулях, где предупреждения могли быть сознательно проигнорированы автором модуля.
\begin{minted}{bash}
$ perl -e 'use warnings; print(5+"a")'
Argument "a" isn't numeric in addition (+) at -e line 1.
\end{minted}

Другой модуль \verb|diagnostics| аналогичен модулю \verb|warnings|, но также выводит подробную инфомацию по каждому предупреждению:
\begin{minted}{bash}
$ perl -e 'use diagnostics; print(5+"a")'
Argument "a" isn't numeric in addition (+) at -e line 1 (#1)
    (W numeric) The indicated string was fed as an argument to an operator
    that expected a numeric value instead.  If you're fortunate the message
    will identify which operator was so unfortunate.
\end{minted}
Это может быть особенно полезно новичкам в \verb|perl|. Использовать же его в production не стоит, так как человек который будет разбираться с предупреждением сможет самостоятельно найти помощь по этой ошибке в интернете.

\subsection{Модули lib и FindBin} %38:30
Модуль \verb|lib| позволяет добавить путь к массиву $@INC$, который содержит директории, в которых будут будет производиться поиск модулей. Вместо того, чтобы вручную добавлять путь с помощью команды \verb|unshift| в блоке \verb|BEGIN|:
\begin{minted}{perl}
use lib qw(/tmp/lib);

BEGIN { unshift(@INC, '/tmp/lib') }
\end{minted}
можно просто воспользоваться этим модулем.

В связке с этим модулем используется модуль \verb|FindBin|, который позволяет сохранить путь к текущему бинарному файлу в некоторой переменной. После этого можно указывать в модуле \verb|lib| путь относительно пути к бинарному файлу:
\begin{minted}{perl}
use FindBin '$Bin';
use lib "$Bin/../lib";
\end{minted}
Обычно так работают standalone-программы.

\subsection{Модуль feature} %40 Опечатка на слайде
Модуль \verb|feautre| позволяет подключить возможность, добавленную в поздних версиях \verb|perl| и которая не была сделана возможностью по умолчанию, например, чтобы избежать конфликта имен. Следующий код подключает функцию \verb|say|, которая отличается от \verb|print| тем, что дополнительно делает перевод строки:
\begin{minted}{perl}
use feature qw(say);

say 'New line follows this';
\end{minted}
Если программист уже определил функцию \verb|say|, то добавление этой функции по умолчанию приведет к конфликту имен.

\subsection{Модуль brignum} %40:13
Модули \verb|bigint| и \verb|bigrat| позволяют отключить встроенное ограничение на длину вычисляемого значения для целых и рациональных чисел соответственно. Например, \verb|bigint| отключает округление при больших значениях целочисленной переменной:
\begin{minted}{perl}
use bignum;
use bigint;
use bigrat;
\end{minted}
\begin{minted}{bash}
$ perl -E 'use bigint; say 500**50'
888178419700125232338905334472656250000000000000000000000000000000000000000000000000000000000000000000000000000000000000000000000000000

$ perl -E 'say 500**50'
8.88178419700125e+134
\end{minted}
Модуль \verb|bignum| подключает оба модуля сразу.

\subsection{Отключение модулей} %43
С помощью \verb|no|, антипода \verb|use|, можно отключить в данный момент не нужные модули. В этом случае вместо метода \verb|import| (в \verb|use|) используется метод \verb|unimport|.
\begin{minted}{perl}
no Local::Module LIST;

# Local::Module->unimport(LIST);
\end{minted}
С помощью \verb|no| можно отключить возможности, которые были добавлены в современных версиях \verb|perl|:
\begin{minted}{perl}
no 5.010;
\end{minted}
Также с помощью \verb|no| можно отключить pragmatic модули, в частности \verb|strict| и \verb|feature|:
\begin{minted}{perl}
no strict;
no feature;
\end{minted}
Обычно эти возможности используются, чтобы локально (в отдельной области видимости) выключить одно из ограничений, накладываемое \verb|strict|, и сделать то, что это ограничение запрещает. После закрывающей фигурной скобки все локально выключенные ограничения вновь будут в силе. Такой подход позволяет использовать потенциально опасные операции только в рамках локальной области видимости и осознанно.


\section{Внутренние механизмы работы perl}
\subsection{Symbol Tables} %42:30
%\subsection{Typeglob}
В \verb|perl| для каждого загруженного пакета создается специальный служебный хеш. Он имеет имя, которое состоит из символа процента, затем имени пакета и двойного двоеточия за ним. Например, если был загружен модуль \verb|Data::Dumper|, то станет доступным хеш \verb|\%Data::Dumper::|.
\begin{minted}{perl}
{
  package Some::Package;
  our $var = 500;
  our @var = (1,2,3);
  our %func = (1 => 2, 3 => 4);
  sub func { return 400 }
}
\end{minted}
Внутри этого хеша можно увидеть так называемую символическую таблицу. Для модуля:
\begin{minted}{perl}
use Data::Dumper;
print Dumper \%Some::Package::;
\end{minted}
соответствующая символическая таблица имеет вид:
\begin{minted}{perl}
$VAR1 = {
          'var' => *Some::Package::var,
          'func' => *Some::Package::func
        };
\end{minted}

% TODO Путается

\subsection{Typeglob} %49 44:30

% TODO Путается

\subsection{Функция caller} %50
Встроенная функция \verb|caller| позволяет получить данные из стека вызовов. Если эта функция была вызвана без параметров, то она вернет название пакета, откуда была вызвана текущая функция, соответствующие имя файла и номер строчки:
\begin{minted}{perl}
# 0         1          2
($package, $filename, $line) = caller();
\end{minted}
В качестве параметра можно указать глубину. В этом случае \verb|caller| вернет гораздо больше информации, в том числе, в каком контексте была вызвана функция и так далее:
\begin{minted}{perl}
(
	$package,    $filename,   $line,
	$subroutine, $hasargs,    $wantarray,
	$evaltext,   $is_require, $hints,
	$bitmask,    $hinthash
) = caller($i);
\end{minted}
\verb|Export| работает именно так: через функцию \verb|caller| узнает имя пакета, откуда он был вызван, а затем с помощью операций над таблицами-символами создаёт нужную функцию в этом пакете.

\subsection{Перехват обращения к несуществующей функции} %51 :48:19
В \verb|perl|, как и во многих других современных интерпретируемых языках, есть способ перехватывать обращения к несуществующим функциям. До того, как будет брошено исключение о том, что запрашиваемой функции нет, будет предпринята попытка вызвать функцию \verb|AUTOLOAD| из этого пакета:
\begin{minted}{perl}
\end{minted}
В переменной пакета $\$AUTOLOAD$ будет лежать имя той функции, которая пыталась быть вызвана. Стоит отметить, что в качестве параметров функции \verb|AUTOLOAD| передаются параметры вызываемой функции.

Это позволяет объявлять не одну функцию, а сразу класс функций. Для тех, кто знаком с интерпретируемыми языками, это механизм уже может быть знаком. Например, в Ruby это называется missing method.

\subsection{Ключевое слово local} %52 50:02
Помимо \verb|my|, \verb|state| и \verb|our|, есть еще похожее на них ключевое слово \verb|local|, которое, однако, не имеет с ними ничего общего. В \verb|perl| существует возможность временно присвоить любой переменной некоторое значение до конца области видимости:
\begin{minted}{perl}

{
  package Test;
  our $x = 123;

  sub bark { print $x }
}

Test::bark(); # 123
{
  local $Test::x = 321;
  Test::bark(); # 321
}
Test::bark(); # 123
\end{minted}
Чаще всего это используется в тех случаях, когда требуется временно поменять поведение и восстановить старое поведение после.

В частности, можно временно изменить значения служебных переменных. Поскольку эти переменные используются внутренними механизмами \verb|perl|, их прежнее значение должно быть возвращено. Примером использование данной возможности является переопределение служебной переменной, которая определяет перенос конца строки, чтобы считывать файлы не построчно, а целиком.

На самом деле с помощью \verb|local| можно переопределять не только переменные, но даже ключи в хеше. Более того, существует конструкция \verb|delete local|, которая удалит ключ, но только локально до конца области видимости, а потом вернет его на место. Возможности \verb|local| безграничны, но рекомендуется не злоупотреблять им, потому что в сложных конструкциях его действие может быть не очевидно:
\begin{minted}{perl}
# localization of values
local $foo;
local (@wid, %get);
local $foo = "flurp";
local @oof = @bar;
local $hash{key} = "val";
delete local $hash{key};
local ($cond ? $v1 : $v2);

# localization of symbols
local *FH;
local *merlyn = *randal;

local *merlyn = 'randal';
local *merlyn = \$randal;
\end{minted}
Подробную справку по ключевому слову \verb|local| можно найти в документации.

%\setcounter{chapter}{3}
\chapter{Unicode и регулярные выражения}
Данные темы, Unicode и регулярные выражения, относятся в большей части не конкретно к языку Perl, а к программированию в целом: данная лекция будет полезна независимо от выбора языка для разработки.

\section{Unicode} %1 0:50
Unicode --- стандарт кодирования символов, позволяющий представить знаки практически всех письменных языков. Каждый символ идентифицируется уникальным кодом, но этот код может быть представлен по-разном в виде различных последовательностей байт.

Unicode Transformation Format (Формат преобразования юникода)~--- способ представления символов Unicode в виде последовательности целых положительных чисел. Наиболее распространенные кодировки:
\begin{itemize}
  \item UTF-8 (8-битный) endianness safe
  \item UTF-16 (16-битный) LE | BE
  \item UTF-32 (32-битный) LE | BE
\end{itemize}
Кодировка utf8 позволяет кодировать символы в следующих диапазонах:
\begin{verbatim}
  Code Points   Bytes: 1st    2nd    3rd    4th

   U+0000..U+007F     00..7F
   U+0080..U+07FF     C2..DF 80..BF
   U+0800..U+0FFF     E0     A0..BF 80..BF
   U+1000..U+CFFF     E1..EC 80..BF 80..BF
   U+D000..U+D7FF     ED     80..9F 80..BF
   U+D800..U+DFFF     utf16 surrogates, not utf8
   U+E000..U+FFFF     EE..EF 80..BF 80..BF
  U+10000..U+3FFFF    F0     90..BF 80..BF 80..BF
  U+40000..U+FFFFF    F1..F3 80..BF 80..BF 80..BF
 U+100000..U+10FFFF   F4     80..8F 80..BF 80..BF
\end{verbatim}
Начальный диапазон совпадает с ASCII (за что ее и полюбили) и в нем длина записываемого символа равна одному байту. Чем дальше расположен код, тем большее число байт необходимо, чтобы его закодировать.

\section{Unicode в perl}  %10 3:42
В perl выделяют следующие понятия:
\begin{itemize}
  \item Символ --- любой unicode-символ:
  \begin{minted}{perl}
    "\x{1}" .. "\x{10FFFF}"
    chr(1)  .. chr(0x10FFFF)
  \end{minted}
  \item Байт (октет) --- символ с кодом до 255:
  \begin{minted}{perl}
    "\x00" .. "\xff"
    "\000" .. "\377"
    chr(0) .. chr(255)
  \end{minted}
\end{itemize}

Обычно создается байтовая строка:
\begin{minted}{perl}
my $bytes = "123";
printf "%vX", $bytes; # 31.32.33
my $bytes = "\001\002\377";
printf "%vX", $bytes; # 1.2.ff
my $bytes = "\xfe\xff";
printf "%vX", $bytes; # fe.ff
\end{minted}
Если подключить определенную директиву, можно создавать строку из символов:
\begin{minted}{perl}
use utf8;
my $string = "Ёлка";#\x{401}\x{43b}\x{43a}\x{430}
printf "%vX", $string; # 401.43B.43A.430
my $string = "\x{0401}";
printf "%vX", $string; # 0401 
\end{minted}
Преобразования между этими двумя представлениями строк это:
\begin{itemize}
    \item \textbf{encoding} --- преобразование текста (строк, символов) в данные (байты, октеты)
    \item \textbf{decoding} -- преобразование данных (байт, октетов) в текст (строки символов)
\end{itemize}
С помощью штатного модуля Encode можно делать такие преобразования:
\begin{minted}{perl}
use Encode qw(encode decode);

my $bin = "\320\201";
printf "%vX", $bin; # D0.81

my $str = decode("utf-8", $bin); # "\x{0401}"
printf "%vX",$str; # 0401

my $bin = encode("utf-8", $str); # "\320\201"
printf "%vX", $bin; # D0.81

my $bytes_dos = "\xf1"; # cp866 ё
printf "%vX", $bytes_dos; # F1
my $chars = decode("cp866",$bytes_dos);
my $bytes_win = encode("cp1251", $chars);
printf "%vX", $bytes_win; # B8

my $to = encode("cp1251",decode("cp866",$from));
from_to($from,"cp866","cp1251"); # inplace
\end{minted}
Для того, чтобы преобразовать одну кодировку к другой, сначала необходимо сделать decode из одной кодировки, а потом encode в другую.

Perl различает строки от октетов с помощью так называемого utf8 flag (введен в версии 5.8). У каждой переменной, которая является строкой, есть внутренний атрибут, который говорит, является ли она строкой или представлением байт. Функция \verb|utf8::is_utf8| возвращает true или false в зависимости от этого.
\begin{minted}[escapeinside=||]{perl}
say utf8::is_utf8("\320\201"); # ''
my $string = decode("utf-8", "\320\201");
say utf8::is_utf8($string); # 1

say utf8::is_utf8("\x{0401}"); # 1
my $octets = encode("utf-8", "\x{0401}");
say utf8::is_utf8($octets); # ''

printf "U+%v04X\n", decode('utf8',"тест");
# U+0442.0435.0441.0442

say utf8::is_utf8("Ё"); # ''

printf "U+%v04X\n", "Ё";
# U+00D0.0081
\end{minted}

Здесь с помощью специальной последовательности \verb|"U+%v04X\n"| (которую нужно передать функции printf) был выведен шестнадцатеричный код остальных символов. Это полезно для отладки.

С помощью директивы use utf8 можно сообщить Perl, что необходимо декодировать весь текст в файле с исходным кодом программы. Если там содержатся unicode-строки, то на них автоматом взводится utf8 flag:
\begin{minted}{perl}
use utf8;

say utf8::is_utf8("\320\201"); # ''

say utf8::is_utf8("\x{0401}"); # 1

say utf8::is_utf8("Ё"); # 1
\end{minted}
С помощью модуля DevelL::Peek можно увидеть внутреннее устройство переменных в perl (о нем подробнее речь пойдет позже):
\begin{minted}{bash}
$ perl -MDevel::Peek -E 'Dump "Ё"'
SV = PV(0x7fad0a006540) at 0x7fad0a02d700
  REFCNT = 1
  FLAGS = (POK,IsCOW,READONLY,PROTECT,pPOK)
  PV = 0x7fad09d024d0 "\320\201"\0
  CUR = 2
  LEN = 10
\end{minted}
Так можно просмотреть содержимое переменной с включенной директивой utf8:
\begin{minted}{bash}
$ perl -MDevel::Peek -Mutf8 -E 'Dump "Ё"'
SV = PV(0x7fbaa8806560) at 0x7fbaa882d790
  REFCNT = 1
  FLAGS = (POK,IsCOW,READONLY,PROTECT,pPOK,UTF8)
  PV = 0x7fbaa8404fe0 "\320\201"\0 [UTF8 "\x{0401}"]
  CUR = 2
  LEN = 10
\end{minted}
У переменной появился флаг utf8.

Существует еще одна особенность: если код символа меньше 255, флаг utf8 не взводится автоматически:
\begin{minted}{bash}
$ perl -MDevel::Peek -E 'Dump "\x{ff}"'
SV = PV(0x7fa153802948) at 0x7fa153005b00
  REFCNT = 1
  FLAGS = (PADTMP,POK,READONLY,pPOK)
  PV = 0x7fa152d06a10 "\377"\0
  CUR = 1
  LEN = 16
\end{minted}
Если код становится больше, utf8 флаг вместе с этим взводится.
\begin{minted}{bash}
$ perl -MDevel::Peek -E 'Dump "\x{100}"'
SV = PV(0x7fcdbc003548) at 0x7fcdbc02c100
  REFCNT = 1
  FLAGS = (PADTMP,POK,READONLY,pPOK,UTF8)
  PV = 0x7fcdbb707110 "\304\200"\0 [UTF8 "\x{100}"]
  CUR = 2
  LEN = 16
\end{minted}
Встроенные функции, в отличие от операторов, полагаются на наличие флага utf8. Они не смотрят на окружения и директивы. Например:\\
\begin{minted}{perl}
my $test = "тест";
say length $test;
say uc $test;
say utf8::is_utf8 $test;
say ord(substr($test,0,1));
printf "%vX", $test;
\end{minted}
\begin{minted}{perl}
#
8
тест
''
209
D1.82.D0.B5.D1.81.D1.82
\end{minted}

В свою очередь:
\begin{minted}{perl}
use utf8;
my $test = "тест";
say length $test;
say uc $test;
say utf8::is_utf8 $test;
say ord(substr($test,0,1));
printf "%vX", $test;
\end{minted}
\begin{minted}{perl}
#
#
4
ТЕСТ
1
1090 # 0x442
442.435.441.442
\end{minted}

Если необходимо передать в качестве входящего аргумента utf8-строку, то чтобы у нее установился utf8 флаг, необходимо либо передать параметр A ключу -C (который отвечает за поведение юникода):
\begin{minted}{bash}
$ perl -CA ...
\end{minted}
Либо установить соответствующее значение для переменной окружения \verb|PERL_UNICODE|:
\begin{minted}{bash}
$ export PERL_UNICODE=A
\end{minted}
Еще один способ сделать то же самое --- написать в программе (декодировать \verb|@ARGV|):
\begin{minted}{perl}
use Encode qw(decode_utf8);
BEGIN {
    @ARGV = map { decode_utf8($_, 1) } @ARGV;
}
\end{minted}
По умолчанию \verb|STDIN|, \verb|STDOUT|, \verb|STDERR| не являются UTF-8. Другими словами, считанный из файла текст не будет строкой, а при записи UTF-8-строки в stdout или stderr будет выведено <<Wide character in print at...>>. В этом случае необходимо указать, что конкретный дескриптор является дескриптором с UTF-8:
\begin{minted}{bash}
$ perl -CS ...
$ export PERL_UNICODE=S
\end{minted}
Также это может быть сделано вручную на любом дескрипторе при помощи дополнительного слоя <<:utf8>> (подробнее про слои будет рассказано позже в курсе):
\begin{minted}{perl}
binmode(STDOUT,':utf8');
open my $f, '<:utf8', 'file.txt';
use open qw(:std :utf8); # auto
\end{minted}
С помощью модуля open можно установить такое поведение в качестве поведения по умолчанию. Сделать весь ввод/вывод в UTF-8 можно следующими способами. С помощью ключа или нескольких ключей:
\begin{minted}{perl}
$ perl -CASD ... | perl -CS -CA -CD ...
\end{minted}
Также можно соответствующим образом установить переменную окружения:
\begin{minted}{perl}
$ export PERL_UNICODE=ASD
\end{minted}
И наконец:
\begin{minted}{perl}
use open qw(:std :utf8);
use Encode qw(decode_utf8);
BEGIN{ @ARGV = map decode_utf8($_, 1),@ARGV; }
\end{minted}
После этого программа будет читать в unicode, писать в unicode и работать внутри себя в unicode. Чтобы в таком режиме работать с бинарными строками (например открыть бинарный файл), можно использовать слой ввода/вывода raw:
\begin{minted}{perl}
binmode($fh,':raw');

binmode(STDOUT,':raw');

open my $f, '<:raw', 'file.bin';

my $socket = accept(...);
binmode($socket,':raw');
\end{minted}
В Perl ставится во главу угла поддержка максимального количества возможностей unicode. Нововведения стараются внедрить как можно быстрее, поэтому среди всех языков программирования наилучшую поддержку unicode обеспечивает именно Perl.

Например, для последовательности на известном или неизвестном Вам языке можно применить фонетическую транслитерацию с помощью модуля Unidecode:
\begin{minted}{perl}
use utf8;
use Text::Unidecode;

say unidecode "\x{5317}\x{4EB0}"; # 北亰
# That prints: Bei Jing

say unidecode "Это тест";
# That prints: Eto tiest
\end{minted}
Будет выведена английская фонетическая транслитерация.

Модуль charnames позволяет работать с полными названиями символов:
\begin{minted}{perl}
use charnames qw(:full :short latin greek);
say "\N{MATHEMATICAL ITALIC SMALL N}"; # 𝑛
say "\N{GREEK CAPITAL LETTER SIGMA}"; # Σ
say "\N{Greek:Sigma}"; # Σ
say "\N{ae}"; # æ
say "\N{epsilon}"; # ε
say "\N{LATIN CAPITAL LETTER A WITH MACRON AND GRAVE}";
$s = "\N{Latin:A WITH MACRON AND GRAVE}";
say $s;  # Ā̀
printf "U+%v04X\n", $s; # U+0100.0300
\end{minted}
Некоторые символы могут быть на самом деле сборными (например Ā̀  состоит из двух символов: символа A и символа, обозначающего диакритический знак). Также можно регистрировать последовательности для символом (торговые марки не вносятся в таблицу симовлов):
\begin{minted}{perl}
use charnames ":alias" => {
    "APPLE LOGO" => 0xF8FF,
};
say "\N{APPLE LOGO}"; # 
\end{minted}

Следующая замечательная особенность unicode заключается в том, для одной буквы может существовать множество вариантов начертаний (case). Для того, чтобы удобно сравнивать строки (не учитывать case), существует специальная форма букв --- foldcase. Начиная с Perl v5.16 доступна функция fc:
\begin{minted}{perl}
use feature "fc"; # perl v5.16+

# sort case-insensitively
my @sorted = sort {
    fc($a) cmp fc($b)
} @list;

# both are true:
fc("tschüß") eq fc("TSCHÜSS")
fc("Σίσυφος") eq fc("ΣΊΣΥΦΟΣ")
\end{minted}
На этом знакомство с unicode в Perl закончено.

\section{Регулярные выражения} %27 20:19
\subsection{Регулярные выражения. Пример <<Credit card numbers>>}
Регулярные выражения ---  одна из самых значимых частей языка perl, которые были с самого начала и с тех пор только развивались. Регулярные выражения из Perl были выделены в отдельной библиотеки <<perl compatible regular expression>> (pcre) для использования в других языках.

Регулярные выражения --- формальный язык поиска и осуществления манипуляций с подстроками в тексте, основанный на использовании метасимволов.

Например следующее регулярное выражение:
\begin{minted}{perl}
^(?:
    4[0-9]{12}(?:[0-9]{3})?          # Visa
|
    5[1-5][0-9]{14}                  # MC
|
    3[47][0-9]{13}                   # AmEx
|
    3(?:0[0-5]|[68][0-9])[0-9]{11}   # Diners
|
    6(?:011|5[0-9]{2})[0-9]{12}      # Discover
|
    (?:2131|1800|35\d{3})\d{11}      # JCB
)$
\end{minted}
находит в любом тексте номера кредитных карт.

\subsection{Операторы в регулярных выражениях} %31 23:03
Самое простое регулярное выражение --- сопоставление, то есть проверка, есть ли в строке подстрока:
\begin{minted}{perl}
"hello" =~ /hell/; # matches
"1+2" =~ /1+2/;  # not, "12" or "112" will match
"1+2" =~ /1\+2/; # matches
"1+2" =~ /\d\+\d/; # matches
"/my/path" =~ m"/path" # match /path

"bat" =~ /[bcr]at/; # matches
"cat" =~ /[bcr]at/; # matches
"rat" =~ /[bcr]at/; # matches
"fat" =~ /[bcr]at/; # not
"at"  =~ /[bcr]at/; # not
\end{minted}
В случае использование разделителей /.../ использование m не обязательно. В остальных случаях необходимо писать m:
\begin{minted}{perl}
my $string = "sample string";

$string =~  /sample/;
$string =~ m/sample/;
$string =~ m(sample);

my @a = $string =~ /sample/; # list of caps
my $a = $string =~ /sample/; # true|false
if ($string =~ /sample/) # also boolean
   { ... }

for (@samples) {
    /sample/;
    # same as $_ =~ /sample/;
    if (m/sample/) { ... }
    # if ($_ =~ /sample/) { ... }
}
\end{minted}
Сопоставление работает как в списковом контексте (возвращает совпадение), так и скалярном (возвращает true или false). Без указания операнда оператор работает над переменной \verb|$_|.

Следующий оператор --- поиск и замена \verb|s///|:
\begin{minted}{perl}
my $string = "sample string";

$string =~ s/sample/item/;
$string =~ s{sample}{item};
$string =~ s{sample}
            (item);

my $count_of_replace =
    $string =~ s{sample}{item}g;

for (@samples) {
    s/sample/item/;
    # $_ =~ /sample/item/;
}
\end{minted}
По умолчанию в скалярном контексте возвращается количество совершенных замен. Без указания операнда оператор работает над переменной \verb|$_|. Оператор поиска и замены модифицирует ту строку, над которой применяется.


Следующий оператор --- транслитерация (\verb|y///|, \verb|tr///|):
\begin{minted}{perl}
my $str = "MiXeD CaSe StRiNg";

# ASCII lowercase;
$str =~ tr/A-Z/a-z/;
# mixed case string

# Change case
my $str = "MiXeD CaSe StRiNg";
$str =~ tr/A-Za-z/a-zA-Z/;
# mIxEd cAsE sTrInG

# ROT-13
$str =~ tr/A-Za-z/N-ZA-Mn-za-m/;
# zVkRq pNfR fGeVaT
\end{minted}
Это специальное выражение относится к регулярным выражениям по синтаксису, но как таковым выражением не является. Этот оператор проходит по всей строке и ищет в ней символы из левой части. Если таковой встречается, он заменяется на соответствующий символ из правой части.


\subsection{Метасимволы, классы символов и квантификаторы} %34 28:49
Полный список метасимволов, которые используются внутри регулярных выражений:
\begin{verbatim}
{ } [ ] ( ) ^
$ . | * + ? \
\end{verbatim}
Чтобы использовать их как символы, их нужно экранировать. Все остальные символы экранировать не нужно.

Классы символов позволяют указать, какие символы должны стоять на конкретной позиции:
\begin{minted}{perl}
[...]      # перечисление
/[abc]/      # "a" или "b" или "c"
/[a-c]/      # то-же самое
/[a-zA-Z]/   # ASCII алфавит

/[bcr]at/    # "bat" или "cat" или "rat"

[^...]     # отрицательное перечисление
/[^abc]/     # что угодно, кроме "a", "b", "c"
/[^a-zA-Z]/  # что угодно, кроме букв ASCII
\end{minted}
Также существуют заранее определенные классы символов:
\begin{verbatim}
`\d` - цифры. не только `[0-9]` # ۰ ۱ ۲ ۳ ۴ ۵
`\s` - пробельные символы `[\ \t\r\n\f]` и др.
`\w` - "буква". `[0-9a-zA-Z_]` и юникод

`\D` - не цифра. `[^\d]`
`\S` - не пробельный символ. `[^\s]`
`\W` - не "буква". `[^\w]`

`\N` - что угодно, кроме "\n"
`.`  - что угодно, кроме "\n" ⃰
`^`  - начало строки ⃰ ⃰
`$`  - конец строки ⃰ ⃰

>∗  поведение меняется в зависимости от модификатора /s
>∗∗ поведение меняется в зависимости от модификатора /m
\end{verbatim}

Квантификатор после символа, символьного класса или группы определяет, сколько раз предшествующее выражение может встречаться:
\begin{itemize}[nosep]
 \item ? --- 0 или 1 ({0,1})
 \item * --- 0 или более ({0,})
 \item + --- 1 или более ({1,})
 \item {x} --- ровно x
 \item {x,y} --- от x до y включительно
 \item {,y} --- от 0 до y включительно
 \item {x,} --- от x до бесконечности*
\end{itemize}
Здесь бесконечность считается равной 32768.

\begin{minted}{perl}
/^1?$/  # "" or "1"
/^a*$/  # "" or "a", "aa", "aaa", ...
/^\d*$/ # "" or "123", "11111111", ...
/^.+$/  # "1" or "abc", not ""

/^\d{4}-\d{2}-\d{2} \d{2}:\d{2}:\d{2}$/
	# "2015-10-14 19:35:01"
\end{minted}

\subsection{Захваты, группы и оглядывания} %39 %36:00
Захваты --- специальные переменные
\begin{verbatim}
> `$1`, `$2`, `$3`, ...
\end{verbatim}
в которые попадает то, что регулярном выражении записано в круглых скобках:
\begin{minted}{perl}
$_ = "foo bar baz";

m/^(\w+)\s+(\w+)\s+(\w+)$/;
# $1 = 'foo';
# $2 = 'bar';
# $3 = 'baz';

m/^(\w(\w+))\s+((\w+))/;
#  1  2        34
# $1 = 'foo';
# $2 = 'oo';
# $3 = 'bar';
# $4 = 'bar';
\end{minted}
Нумерация переменных выполняется в той последовательности, в какой появляется открывающая скобка соответствующего блока.

Круглые скобки являются самой простой захватывающей группой.
\begin{itemize}[nosep]
  \item (...) --- захватывающая группа
  \item (?:...) --- незахватывающая группа
\end{itemize}

\begin{minted}{perl}
"a" =~ /^(?:a|b|cd)$/;   # match
"b" =~ /^(?:a|b|cd)$/;   # match
say $1; # undef
"ax" =~ /^(?:a|b|cd)$/;  # no match

"a" =~ /^(a|b|cd)$/;   # match
say $1; # a
"b" =~ /^(a|b|cd)$/;   # match
say $1; # b
"ax" =~ /^(a|b|cd)$/;  # no match
say $1; # undef
\end{minted}

\begin{itemize}[nosep]
  \item (?<name>...) или (?'name'...) --- захватывающая именованная группа.
\end{itemize}
Значения соответствующих захватов попадают в специальный хеш \verb|$+|:
\begin{minted}{perl}
"abc" =~ /^(.)(.)/;
say "first: $1; second: $2";
# first: a; second: b

"abc" =~ /^(?<first>.)(?<second>.)/;
say "first: $+{first}; second: $+{second}";
# first: a; second: b
\end{minted}
При этом к именованным захватам все еще можно обратиться по их номеру.

Существую также так называемые оглядывания (look around assertions), которые позволяют <<просматривать>> окружающий текст, но не включать его в захват (то есть имеют нулевую длину).
\begin{itemize}[nosep]
  \item (?=...) --- 0W+ вперёд
  \item (?!...) --- 0W- вперёд
  \item (?<=...) --- 0W+ назад
  \item (?<!...) --- 0W- назад
\end{itemize}
Например:
\begin{minted}{perl}
$_ = "foo bar baz";

say s{(\w+)(?=\s+)}{$1,}r; # foo, bar, baz
say s{(\s+)(?!bar)}{_}r; # foo bar_baz

say s{(?<= )(\w+)}{:$1}rg; # foo :bar :baz
say s{(?<! )(\w\w\w)}{[$1]}rg; # [foo] bar baz
\end{minted}
Заглядывать назад можно только на фиксированную длину.

\subsection{Модификаторы} %42 44:35
Модификатор, если приписать его к регулярному выражению, меняет его поведение:
\begin{itemize}
  \item Модификатор \verb|/s| (single line) приводит к тому, что в класс символов \verb|.| могут попадать переводы строк:
\begin{minted}{perl}
"\n" =~ /^.$/; # no match
"\n" =~ /^.$/s; # match

my $s = "line1\nline2\n";

$s =~ /line1.line2/; # no match
$s =~ /line1.line2/s; # match
\end{minted}
  \item Модификатор \verb|/m| (multiline) меняет поведение крышечки и доллара. Теперь: \verb|^|~--- начало каждой строки,  \verb|$|~--- конец каждой строки (до \verb|\n|). Например:
\begin{minted}{perl}
my $s = "sample\nstring";

$s =~ /^(.+)$/;    # no match
$s =~ /^(.+)$/m;   # "sample"
$s =~ /^(.+)$/ms;  # "sample\nstring"

$s =~ /^string/;   # no match
$s =~ /^string/m;  # matches "string"
\end{minted}
  \item Модификатор \verb|/i| (case insensitive) включает case-insensitive режим с учетом особенностей юникода:
\begin{minted}{perl}
my $s = "sample\nstring";

$s =~ /SAMPLE/;    # no match
$s =~ /SAMPLE/i;   # "sample"

# Unicode!

"tschüß" =~ /TSCHÜSS/i    # match. ß ↔ SS
"Σίσυφος" =~ /ΣΊΣΥΦΟΣ/i   # match. Σ ↔ σ ↔ ς
\end{minted}

\item Модификатор \verb|/x| (eXtended regexp) включает расширенный режим, в котором перестают интерпретироваться пробельные символы, а также появляется возможность добавлять комментарии:
\begin{minted}{perl}
$hexdig =~ m{
    ^ # begin of string
    (?:
        [0-9] # it could be digit
        |     # or
        [a-f] # letters from a to f
    )+ # several times
    $ # end of string
}sx;
\end{minted}

\item Модификатор \verb|/g| (global) приводит к тому, что при поиске и замене поиск и замена будет выполняться до тех пор, пока это возможно. В match в списковом контексте модификатор приводит к тому, что возвращается не первое попадание, а все.
\begin{minted}{perl}
my $s = "aaaa";
$s =~ s/a/b/;  # "baaa"
$s =~ s/a/b/g; # "bbbb"

@a = $a =~ /(.)/; # ('a')
@a = $a =~ /(.)/g; # ('a','a','a','a')

my $string = '~!@#$%^&*()';
$string =~ s{(.)}{\\$1}g;
# \~\!\@\#\$\%\^\&\*\(\)
\end{minted}

\item Модификатор \verb|/e| (eval) приводят к тому, что в регулярном выражении правая часть исполняется как код Perl. Следующее регулярное выражение позволяет преобразовать шестнадцатиричные числа в их десятичное значение:
\begin{minted}{perl}
my $string = '~!@#$%^&*()';

$string =~ s{(.)}{
    sprintf("U+%v04x;",$1)
}ge;
#U+007e;U+0021;U+0040;U+0023;U+0024;U+0025;
#U+005e;U+0026;U+002a;U+0028;U+0029;

my $nums = "0x123 123 0xff";
$nums =~ s{0x([\da-f]+)}{ hex($1) }ge;
say $nums; # 291 123 255
\end{minted}
Также существуют модификаторы \verb|/ee| (double eval), \verb|/eee| и так далее. 

\item Модификатор \verb|/a| (и \verb|/aa|) (ASCII-safe) меняет поведение \verb|\d|, \verb|\s|, \verb|\w| так, чтобы они подходили только под диапазон ASCII. \verb|/aa| приводит к тому, что все спец. символы, которые имеют представление в виде обычных символов, перестанут совпадать с ними.
\begin{minted}{perl}
use utf8;
use charnames ':full';
my $nums = "०१२३";
$nums =~ /\d/; # match
$nums =~ /\d/a; # no match

my $str = "\N{KELVIN SIGN}";
say $str =~ /K/i; # match
say $str =~ /K/ai; # match
say $str =~ /K/aai; # no match
\end{minted}

\end{itemize}

\subsection{Квантификаторы и жадность} % 53 1:02:32
По умолчанию в регулярном выражении квантификатор является жадным и старается захватить подстроку наибольшего размера. Для управления жадностью существует два флага квантификатора:
\begin{itemize}
  \item \verb|?| --- сделать нежадным (захватывать строку минимального размера).
  \item \verb|+| --- запрещает откат.
\end{itemize}

\begin{minted}{perl}
say "bc"    =~ /^(a*)b/;   # match, ""
say "abc"   =~ /^(a*)b/;   # match, "a"
say "aabc"  =~ /^(a*)b/;   # match, "aa"
say "aaabc" =~ /^(a*)b/;   # match, "aaa"

say "aaabc" =~ /^(a*)/;    # match, "aaa"
say "aaabc" =~ /^(a*?)/;   # match, ""
say "aaabc" =~ /^(a*?)a/;  # match, ""
say "aaabc" =~ /^(a*?)ab/; # match, "aa"

say "aaabc" =~ /^(a*+)/;   # match, "aaa"
say "aaabc" =~ /^(a*+)b/;  # match, "aaa"
say "aaabc" =~ /^(a*+)ab/; # no match

say "bc"    =~ /^(a+)b/;   # no match
say "abc"   =~ /^(a+)b/;   # match, "a"
say "aabc"  =~ /^(a+)b/;   # match, "aa"
say "aaabc" =~ /^(a+)b/;   # match, "aaa"

say "aaabc" =~ /^(a+)/;    # match, "aaa"
say "aaabc" =~ /^(a+?)/;   # match, "a"
say "aaabc" =~ /^(a+?)a/;  # match, "a"
say "aaabc" =~ /^(a+?)ab/; # match, "aa"

say "aaabc" =~ /^(a++)/;   # match, "aaa"
say "aaabc" =~ /^(a++)b/;  # match, "aaa"
say "aaabc" =~ /^(a++)ab/; # no match

say "bc"    =~ /^(a{1,2})b/; # no match
say "abc"   =~ /^(a{1,2})b/; # match, "a"
say "aabc"  =~ /^(a{1,2})b/; # match, "aa"
say "aaabc" =~ /^(a{1,2})b/; # no match

say "aaabc"  =~ /^(a{1,2})a/;  # match "aa"
say "aaabc"  =~ /^(a{1,2}?)a/; # match "a"
say "aabc"   =~ /^(a{1,2}?)b/; # match "aa"
\end{minted}

\subsection{Backreferencing и Global match}
Внутри регулярного выражения можно сослаться с помощью \verb|\gN|, \verb|\N| или \verb|\g{-N}| на предыдущий захват:
\begin{minted}{perl}
for (
    q{some with "quoted value" string},
    q{some with 'quoted " value' string},
) {
    say $2 if m{(["'])([^\g1]*)\g1};
}
# quoted value
# quoted " value

for ('e66e', 'f99f', 'z87z' ) {
    say $1 if m{(([a-z])(\d)\g{-1}\g{-2})}x;
}
#e66e
#f99f
\end{minted}

Помимо стандартного поведения Global match для того, чтобы заменить все или найти все. Обычный match в списковом контексте возвращает все совпадения, а в скалярном контексте, а условия в цикле --- это скалярный контекст, он делает следующую вещь.

У любой переменной строкового Perl есть внутреннее свойство --- позиция (можно прочитать и установить с помощью функции \verb|pos|), которое учитывается при исполнении регулярных выражений. Обычный match в скалярном контексте с модификатором g возвращает ровно одно совпадение, но оставляет позицию в той точке, в которой регулярное выражение остановилось. Следующая итерация данного регулярного выражения будет работать уже не с начала строки, а с этой позиции:
\begin{minted}{perl}
$_ = "abcd";
while (/(.)/g) {
    say $1, " ", pos($_);
    # a 1
    # b 2
    # c 3
    # d 4
}
say $1, " ", pos($_);
# undef, undef
\end{minted}
Когда регулярное выражение неуспешно, т.е. дошло до конца строки, значение pos сбрасывается в исходное значение undef. При следующем применении регулярного выражения поиск опять начнется с начала строки.

Для того, чтобы позиция не сбрасывалась, существует ключ \verb|/c|:
\begin{minted}{perl}
$_ = "abcd";
while (/(.)/gc) {
    say $1, " ", pos($_);
    # a 1
    # b 2
    # c 3
    # d 4
}
say $1, " ", pos($_);
# undef, 4
\end{minted}
В этом случае, если регулярное выражение неуспешно, позиция не меняется.

Еще один пример:
\begin{minted}{perl}
$_ = "abcdxcdcd";
while (/\G(.)/gc) {
    my $key = $1;
    my $pos = pos($_);
    if (/\Gcd/gc) {
        say "the key before cd is $key at $pos";
    } else {
        say "no cd next after $key";
    }
}
# no cd next after a
# the key before cd is b at 2
# the key before cd is x at 5
# no cd next after c
# no cd next after d
\end{minted}
Здесь /G интерпретируется как позиция, на которую указывает pos.

\subsection{Классы символов Unicode}
В unicode существую категории символов.
\begin{minted}{perl}
`\p{Category}` - совпадение с категорией
`\P{Category}` - исключение категории
`\N{SYMBOL NAME}` - точное имя (см. charnames)
\end{minted}

\begin{minted}{perl}
"UPPER" =~ /\p{IsUpper}/; # match
"UPPER" =~ /\P{IsUpper}/; # no match
"UPPER" =~ /\p{IsLower}/; # no match
"UPPER" =~ /\P{IsLower}/; # match
\end{minted}
Следующий код исключает все кавычки из текста, заменяя их на пробелы:
\begin{minted}{perl}
say q{«string"with'quotes»} =~
    s/\p{Quotation Mark}+/ /rg;
# ' string with quotes '
\end{minted}

Регулярные выражения в Perl, строго говоря, уже являются нерегулярными. Термин <<регулярное выражение>> пришел из математики. Такие выражения не должны быть рекурсивными, однако в Perl регулярные выражения были расширены и теперь можно использовать рекурсивные выражения. Название <<регулярное выражение>> закрепилось исторически.

\section{Отладка регулярных выражений}
В модуле re есть режим отладки:
\begin{minted}{perl}
use re 'debug';
\end{minted}
или же из командной строки:
\begin{minted}{bash}
perl -Mre=debug -E '"aaabc"   =~ /^(a{1,2}?)ab/;'
\end{minted}
При встрече регулярного выражения в этом случае будет выводиться отладочная информация:
\begin{minted}{bash}
Compiling REx "^(a{1,2}?)ab"
Final program:
   1: BOL (2)             # Beginning of line
   2: OPEN1 (4)           # Open group 1
   4:   MINMOD (5)        # Nongreedy (?)
   5:   CURLY {1,2} (9)   # Quantifier {}
   7:     EXACT <a> (0)
   9: CLOSE1 (11)
  11: EXACT <ab> (13)
  13: END (0)
\end{minted}
Регулярное выражение компилируется в псевдопрограмму на более-менее интуитивно понятном языке. После того, как регулярное выражение используется, выводится ход исполнения этой программы:
\begin{minted}{bash}
Guessed: match at offset 0
Matching REx "^(a{1,2}?)ab" against "aaabc"
   0 <> <aaabc>              |  1:BOL(2)
   0 <> <aaabc>              |  2:OPEN1(4)
   0 <> <aaabc>              |  4:MINMOD(5)
   0 <> <aaabc>              |  5:CURLY {1,2}(9)
                  EXACT <a> can match 1 times out of 1...
   1 <a> <aabc>              |  9:  CLOSE1(11)
   1 <a> <aabc>              | 11:  EXACT <ab>(13)
                    failed...
                  EXACT <a> can match 1 times out of 1...
   2 <aa> <abc>              |  9:  CLOSE1(11)
   2 <aa> <abc>              | 11:  EXACT <ab>(13)
   4 <aaab> <c>              | 13:  END(0)
Match successful!
\end{minted}
Это позволяет отлаживать регулярные выражения

\setcounter{chapter}{4}
\chapter{Общение с внешним миром}
Эта лекция посвящена работе с файлами, которые могут находиться как на локальном компьютере, так и на удаленном, работе с другими процессами (для каждого отдельного процесса все остальные являются внешним миром) и так далее.

\section{Работа с файлами}
\subsection{Открытие текстовых файлов}
Работа с файлами в языке perl является практически прямым binding'ом библиотеки из Си и, как следствие, происходит похожим образом. В отличие от Си, где работа с файлами происходит непосредственно с файловыми дескрипторами, в perl существует особый тип FILEHANDLE (<<файловый манипулятор>>), который представляет возможность работы с файлами и работает с файловыми дескрипторами внутри себя.

При запуске Perl открываются 3 файловых манипулятора (для работы со стандартными потоками ввода вывода), которые прямо внутри языка называются следующими BAREWORD'ами:
\begin{description}
  \item[STDIN] --- для работы со стандартным вводом
  \item[STDOUT] --- для работы со стандартным выводом
  \item[STDERR] --- для работы с потоком для вывода диагностических и отладочных сообщений
\end{description}
Сейчас эти BAREWORD'ы остались только для обратной совместимости. Правилом хорошего тона сейчас является использование обычных переменных (например \verb|my $fh|) для хранения файловых манипуляторов, которые можно передавать в функции, которые работают с файлами (например, open или readline).

Ошибка при открытии файла с помощью \verb|open| не является фатальной и не вызывает исключений: чтобы проверить, удалось ли открыть файл, необходимо смотреть на возвращаемое функцией \verb|open| значение. Если открытие файла прошло успешно, возвращается истина (единица), если же нет --- ложь (пустая строка). Поэтому часто используется конструкция \verb|open or die|, при исполнении которой в случае невозможности открыть файл вызывается исключение:
\begin{minted}{perl}
open(my $fh, "<", "input.txt")
  or die "Can't open < input.txt: $!";
\end{minted}
Функция \verb|close| используется тогда, когда файловый манипулятор не нужен и его можно закрыть. Теоретически, perl сам закрывает все файловые манипуляторы при завершении программы. Однако некоторые программы работают очень долго, а значит нужно внимательно следить, чтобы ненужные файлы были закрыты. Это, в частности обусловлено тем, что существуют ограничения со стороны операционной системы на количество одновременно открытых файлов.

В perl существуют следующие режимы работы с файлами:
\begin{itemize}[nosep]
  \item Режим чтения (\verb|<|) позволяет только прочитать содержимое файла.
  \item Режим записи (\verb|>|) стирает содержимое файла и открывает файл на запись.
  \item Режим дозаписи (\verb|>>|) позволяет дописать в конец файла.
  \item Режим записи и чтения (\verb|+>|) стирает содержимое файла, позволяет писать в него и читать то, что в этот файл было записано.
  \item Режим чтения и записи (\verb|+<|) не затирает его содержимое, ставит указатель на начало файла (если сразу начать что-то писать, то новая информация затрет существующие данные), позволяет перемещать указатель, записывать и считывать данные.
\end{itemize}
Каждый режим подходит для своей ситуации и их не стоит путать, чтобы избежать ошибок и не потерять важные данные. Эти режимы есть отражение режимов работы с файлами у функции \verb|fopen| в СИ (\verb|r,r+,w,w+,a|).

При открытии файла в perl можно указать кодировку, в которой записан этот файл, и в этом случае perl будет на лету делать необходимые преобразования. Это позволяет считывать файл не побайтово, а посимвольно, что удобно при работе с текстовыми данными. Также правильное указание кодировки необходимо для нормальной работы регулярных выражений, сравнений и сортировки, так как получившейся строке будет сразу выставлен utf-флаг.

Простейший пример перекодировки файла из CP-1251 в UTF-8:
\begin{minted}{perl}
open(my $fh_cp,'<:encoding(CP-1251)','cp1251.txt');
open(my $fh_utf, '>:encoding(UTF-8)', 'utf8.txt');
while( <$fh_cp> ){
    print $fh_utf $_;
}
close($fh_utf);
close($fh_cp);
\end{minted}

\subsection{Чтение и запись текстовых файлов}
Чтение из файлового манипулятора происходит с помощью оператора <<ромбик>> $<>$:
\begin{minted}{perl}
$input= <>
$line = <$handle>
@lines = <$handle>
\end{minted}
Если между угловыми скобками не указано имя файлового манипулятора, чтение будет производиться из стандартного ввода. При чтении файла в скалярном контексте оператор <<ромбик>> возвращает одну строку файла, а при чтении в списковом контексте --- массив из всех строк. Возможностью считывать массив из всех строк файла нужно пользоваться очень аккуратно, так как открытие большого или очень большого файла может привести к нехватке памяти и краху программы.

Запись и файловый манипулятор делается с помощью \verb|print|:
\begin{minted}{perl}
print $var;
print $fh $var;
\end{minted}
Если файловый манипулятор не указан в качестве первого аргумента \verb|print|, запись идет в стандартный вывод. Также \verb|print| поддерживает передачу BAREWORD в качестве файлового манипулятора:
\begin{minted}{perl}
print STDERR $var;
\end{minted}
В том числе, если BAREWORD был объявлен файловым манипулятором с помощью функции \verb|open|. Этот синтаксис поддерживается только в целях обратной совместимости, встречается в старых модулях и не должен быть использован при наприсании нового кода.

\subsection{Сохранение данных непосредственно в файле модуля}
Данные можно сохранять непосредственно в файле .pm модуля:
\begin{minted}{perl}
package mypkg;

sub read_my_data {
    my @lines = <DATA>;
    return \@lines;
}

1;
__DATA__
This is data from pm file
\end{minted}
В этом случае интерпретатор perl будет интерпретировать как код на perl только часть до конструкции \verb|__DATA__ |. Содержимое после этой конструкции будет доступно как файловый манипулятор \verb|DATA|:
\begin{minted}{perl}
my $line = <DATA>
\end{minted}
Стоит отметить, что вызывать \verb|open| не нужно.

Сохранять данные вместе с кодом часто бывает нужно при написании шаблонизаторов. Например, если нужно сгенерировать письмо пользователям по шаблону, его, чтобы не держать отдельный файл и/или не потерять его, можно как раз хранить в DATA.

\subsection{Чтение и запись бинарных файлов}
Бинарные файлы используются часто в тех случаях, когда текстовые данные занимают много места. Чтобы указать, что из файлового манипулятора необходимо читать по байтам, нужно выполнить:
\begin{minted}{perl}
binmode($fh);  # open with :raw      # для работы с двоичными данными
\end{minted}
Для работы с бинарными файлами perl предоставляет следующие возможности:
\begin{minted}{perl}
syswrite($fh, $data, length($data)); # небуферизированная запись
sysread($fh, $data, $data_size);     # прямой вызов системного read
read($fh, $data, $data_size);        # чтение двоичных данных
eof($fh);                            # проверка на отсутствие данных
\end{minted}
Важно отметить, что \verb|syswrite| и \verb|sysread| это небуферизированные запись и чтение, а команда \verb|read| обращается к определенному модулю perl, который сам решает, сколько нужно прочитать у операционной системы, какую часть выдать, а какую оставить в буфере (этот буфер находится внутри perl). Поэтому полностью быть увереным, что запись или чтение были выполнены полностью (все данные записаны и все считаны) можно только, если использовались прямые системные вызовы \verb|syswrite| и \verb|sysread|.

Пример, демонстрирующий работу с бинарными данными:
\begin{minted}{perl}
use strict;
use Digest::MD5 qw/md5_hex/;

$\ = "\n";
my $data = '';
my $data_size = 1024; # Размер записи в файле

open(my $fh, '<:raw', 'data.bin') or die $!;

while(!eof($fh)){
    read($fh, $data, $data_size) == $data_size
        or die("Неверный размер");
    print md5_hex($data);
}

close($fh);
\end{minted}
Здесь показано, как сразу открыть файл в бинарном режиме. Эта программа последовательно читает записи из бинарного файла (размер каждой записи составляет 1 килобайт) и выводит контрольную сумму MD5 для прочитанного фрагмента. Если размер файла неверный, программа сообщает об этом и прекращает работу.

При работе с файлом в бинарном режиме доступен произвольный доступ к файлу, то есть не обязательно прочитывать весь файл, если известно где хранятся нужные данные. Это очень часто используется, когда размер каждой записи в файле фиксирован, так как позволяет быстро переходить к нужной. Каждый раз, когда что-то прочитано из файла, операционная система запоминает, в каком месте остановилось чтение из этого файла. В perl доступны следующие команды для обеспечения произвольного доступа:
\begin{minted}{perl}
seek( $fh, $len, $type);   # позиционирование
tell($fh);                 # текущая позиция
\end{minted}
Команда \verb|seek| перемещает курсор на \verb|$len| в файловом манипуляторе \verb|$fh| от начала файла (если \verb|$type| равен 0), текущей позиции (если \verb|$type| равен 1) или его конца (если \verb|$type| равен 2).

Команда \verb|tell| возвращает текущую позицию курсора (в байтах) для данного файлового манипулятора. Если ее сохранить, то позже можно будет вернуться в это же место файла с помощью \verb|tell|.

\subsection{Операции проверки файлов}
Существуют операторы, которые позволяют проверять, можно ли открыть файл, записать в него и так далее. Считается хорошим тоном всегда проверять, может ли файл быть открыт до вызова команды \verb|open|. Это связано с тем, что в perl очень ограниченно поддерживаются исключения и недоступна обработка исключений. Операторы проверки файлов начинаются с дефиса (что может сбивать с толку), а за самим оператором следует файловый манипулятор:
\begin{verbatim}[nosep]
  -r  Файл доступен для чтения текущему пользователю
  -w  Файл доступен для записи текущему пользователю
  -x  Файл доступен для выполнения текущим пользователем
  -o  Файл принадлежит текущему пользователю

  -e  Файл существует
  -z  Файл пуст (имеет нулевой размер)
  -s  Файл непуст (возвращает размер в байтах)

  -f  Файл является простым файлом
  -d  Файл является папкой
  -l  Файл является символической ссылкой (всегда false,
            если не поддерживается файловой системой)
  -p  Файл является именованным каналом (FIFO)
  -S  Файл является сокетом.
\end{verbatim}
Обычно они используются следующим образом:
\begin{minted}{perl}
my $fname = 'file.txt';
my $fh;
open($fh, '<', $fname) if
    -e $fname and -f $fname and
    -r $fname and !-z $fname;
\end{minted}
Писать в этом случае \verb|or die| имеет смысл только, если были сделаны не все проверки (если все проверки были сделаны, \verb|or die| не выполнится никогда). Более того \verb|open| внутри себя делает все проверки, кроме проверки длины файла. К слову, проверку длины файла можно использовать, чтобы выдать пользователю информативное сообщение об ошибке, если открытый для чтения файл имеет нулевую длину.

\subsection{Работа с файловой системой}
Существуют следующие команды для работы с файлами:
\begin{verbatim}
  rename     Команда для переименования файла
  unlink     Команда для удаление файла
  truncate   Команда для очистка файла
  stat       Команда для получения информация о времени доступе к файлу
  utime      Команда для модификации информация о времени доступе к файлу
             (поведение команды utime зависит от используемой ОС)
\end{verbatim}

Существуют следующие команды для работы с директориями:
\begin{minted}{perl}
mkdir 'dir_name',   0755; # Создание директории
rmdir 'dir_name';         # Удалении директории
chdir 'dir_name';         # Измерение текущего рабочего каталога
\end{minted}
По умолчанию рабочим каталогом становится каталог, откуда был запущен бинарный файл perl. Если требуется написать приложение, поведение которого не зависит от места запуска, есть два варианта. Первый вариант --- всегда использовать полные пути, но такой вариант не всегда является подходящим, так как полные пути зависят от того, как была сконфигурирована операционная система. Поэтому лучшим вариантом будет определять положение бинарного файла приложения, а после устанавливать корневой каталог относительно него.

Удобный способ создать группу директорий, которых нет в файловой системе, это использовать \verb|make_path|:
\begin{minted}{perl}
use File::Path qw/make_path/;
make_path( '/full/path/to/dir',
    owner => 'user',
    group => 'group',
    mode => 0755);
\end{minted}
Функция \verb|make_path| может принимать широкий набор параметров.

Для чтения и работы с директориями существуют команды, аналогичные файловым командам:
\begin{verbatim}
  opendir    Открыть указанную папку
  readdir    Последовательно читать файлы в папке
  teldir     Сообщить текущую позицию внутри каталога
  seekdir    Перейти на заданную позицию в каталоге
  closedir   Завершить работу с этой папкой
\end{verbatim}
В качестве примера можно привести следующий код:
\begin{minted}{perl}
opendir(my $dh, 'path_to_dir') or die $!;

my $pos;

while(my $fname = readdir $dh){
    print $fname;
    $pos = telldir $dh if $fname = 'data.bin';
}

if ($pos){
    seekdir($dh, $pos);
    while(my $fname = readdir $dh){
        print "Second iter: $fname";
    }
}

closedir($dh);
\end{minted}


\section{Perl I/O backend} % 36:08
\subsection{Perl I/O}
В perl используются файловые манипуляторы, а не файловые дескрипторы (которые предоставляет непосредственно операционная система), поскольку в нем реализована система ввода-вывода perlio с поддержкой слоев.
Существуют, например, следующие слои:
\begin{verbatim}
  :unix      использование pread/pwrite
  :stdio     использование fread, fwrite, fseek/ftell
  :perlio    перл буфер для быстрого доступа к данным после чтения
             и минимизации копирования (readline/<>)
  :crlf      преобразование перевода строки в формат данной ОС
  :utf8      работа в UTF-8
  :encoding  перекодировка содержимого файла
  :bytes     работа с однобайтовыми кодировками
  :raw       binmode()
  :pop       псевдослой, который позволяет убрать из цепочки верхний слой
\end{verbatim}


\subsection{Буферизация ввода-вывода в Perl I/O}
Такая система была встроена в perl по причине того, что он является консольным языком. Часто было необходимо прочитывать последовательно из файла небольшие фрагменты. Использовать для этого системные вызовы неразумно, так как система всегда читает по блокам (например по 1кб), а затем выделяет из блока требуемый фрагмент. В итоге один и тот же блок был бы многократно прочитан, что сильно бы сказывалось на производительности. Система perlio включает в себя слой-буфер \verb|:perlio|, который делает следующее: при первом запросе фрагмента какого-либо блока слоем делается системный вызов на чтение всего блока, слой передает нужный фрагмент, а блок --- буферизуется. При последующих запросах фрагментов из этого же блока, системный вызов уже не производится, а возвращается значение из буфера.

Чтобы продемонстрировать выигрыш производительности от использования такой системы ввода вывода, можно провести тестирование производительности с помощью следующей программы:
\begin{minted}{perl}
use strict;
use PerlIO;
use Time::HiRes qw/gettimeofday/;

my $i=0;
my $start_time = gettimeofday();

while(<>){$i++}

print "Layers: ".join(',', PerlIO::get_layers(STDIN)).'; ';
print "lines: $i; time: ".(gettimeofday() - $start_time).$/;
\end{minted}
Тестирование на файле из почти трех миллиона строк дало следующий результат:
\begin{verbatim}
Layers: unix,perlio; lines: 2932894; time: 0.410083055496216
Layers: stdio; lines: 2932894; time: 3.00101494789124
Layers: unix; lines: 2932894; time: 33.2629461288452
\end{verbatim}
При использовании системы perlio буферизация происходит на уровне perlio и время чтения всех строк составило половину секунды. При использовании stdio буферизация происходит на уровне библиотеки языка Си (поскольку буферизация в Си не учитывает размер блока) и результат несколько хуже --- 3 секунды. Без использования буферизации результат составил 33 секунды.

\subsection{Подключение слоя из внешних библиотек}
Слой \verb|:via| представляет собой универсальный интерфейс, предоставляющий возможность подключения слоя из внешних библиотек. Например, \verb|PerlIO::via::gzip|, позволяет работать с архивами, архивируя и разархивируя данные на лету:
\begin{minted}{perl}
open( $cfh, ">:via(gzip)", 'stdout.gz' );
print $cfh @stuff;

open( $fh, "<:via(gzip)", "stuff.gz" );
while (<$fh>) {
...
}
\end{minted}
На CPAN представлено множеством слоев, которые можно подключить через \verb|:via|, в том числе реализующие временные файлы в оперативной памяти, а также слои, позволяющие автоматически парсить JSON или XML, и так далее. Безусловно, существует возможность самостоятельно написать новый слой, если потребуется.

\section{Взаимодействие процессов.} %19 46:21
При написании параллельных приложений важно обеспечить эффективное взаимодействие между процессами. Процессы могут обмениваться данными и информацией о состоянии.

\subsection{Исполнение команды командной строки}
Простейший вид взаимодействия процессов --- исполнение команды терминала, которая поддерживается в данной операционной системе, и сохранение результата. Это можно сделать с помощью backticks-оператора (\verb|'...'|), функции \verb|system()| или \verb|open()|:
\begin{minted}{perl}
my $out = `ls -l`;               # Блокирующий вызов, возвращает содержимое STDOUT в виде одной большой строковой переменной.
my @out = `ls -l`;               # Блокирующий вызов, возвращает содержимое STDOUT в виде массива строк.
system('ls -l');                 # Блокирующий вызов, возвращает только код завершения
open(my $out, '-|', 'ls', '-l'); # Не-блокирующий вызов. Возвращает содержимое STDOUT через pipe. Данные доступны через файловый манипулятор $out.
\end{minted}
В списковом контексте backticks-оператор возвращает результат из STDOUT в виде массива строк.

\subsection{Вызовы fork и exec}
При написании программы, которая будет работать в несколько процессов, неудобно писать для каждого процесса свою отдельную программу. Удобнее в определенных местах программы добавить точки ветвления с помощью системного вызова fork, который создает дочерний процесс, который является практически полной, независимой копией родительского процесса.

Вызов fork является очень эффективным системным вызовом, так как операционной системе нужно только добавить запись в своих внутренних структурах о том, что появился еще один процесс, память процесса при этом не копируется. Независимость исполнения обеспечивается механизмом copy-on-write: как только в одном из процессов модифицируется какой-нибудь общий блок памяти, он сначала копируется, чтобы память другого процесса в результате не изменилась.

Другая функция exec подменяет текущий процесс каким-то другим процессом. Эта операция тоже эффективная, так как операционной системе не нужно модифицировать свою таблицу процессов, очищать старую память процесса и заводить новую. Эта функция часто используется для создания инсталляторов на perl, которые после сбора параметров от пользователя <<превращаются>> в процесс разархивирования с нужными параметрами.

Поскольку дочерний и родительский процессы после fork выполняются независимо, то они и не взаимодействуют друг с другом. Взаимодействие процессов можно организовать с помощью \verb|pipe|. В результате организуется односторонний буфер так, что один процесс сможет в него только писать, а второй --- из этого же манипулятора что-то вычитать. Пример, как можно создать pipe между двумя процессами:
\begin{minted}{perl}
use strict;
use POSIX qw(:sys_wait_h);
$|=1;

my ($r, $w);
pipe($r, $w);
if(my $pid = fork()){
    close($r);
    print $w $_ for 1..5;
    close($w);
    waitpid($pid, 0);
}
else {
    die "Cannot fork $!" unless defined $pid;
    close($w);
    while(<$r>){ print $_ }
    close($r);
    exit;
}
\end{minted}
При выполнении системного вызова \verb|fork|, в родительском процессе в качестве результата его выполнения возвращается pid дочернего процесса, а в дочернем --- 0. Если операция не прошла успешно, возвращается значение undef. Команда \verb|waitpid| позволяет ждать, пока будет закончено выполнение другого процесса.

\subsection{Обработка сигналов. Уборка процессов-зомби}
Зомби --- процесс, который завершился, но код его завершения не был вычитан. Такой процесс висит в списке процессов операционной системы с пометкой <<z>>. Это сделано для того, чтобы не потерять информацию о коде завершения, если процесс, для которого предназначена эта информация, не может в данный момент ее прочитать.

Чтобы зомби не было, необходимо всегда определять параметр \verb|SIG{CHLD}|:
\begin{minted}{perl}
$SIG{CHLD} = sub {
  while( my $pid = waitpid(-1, WNOHANG)){

    last if $pid == -1;

    if( WIFEXITED($?) ){
      my $status = $? >> 8;
      print "$pid exit with status $status $/";
    }
    else {  print "Process $pid sleep $/"  }
  }
};
\end{minted}
Здесь \verb|WIFEXITED| --- истина, если процесс завершился, \verb|WEXITSTATUS| будет содержать код возврата (если \verb|WIFEXITED| --- истина). Если же процесс был только остановлен, то \verb|WIFSIGNALED| --- истина, а \verb|WTERMSIG| будет содержать номер сигнала, остановившего процесс. В данном примере функции \verb|waitpid| в качестве первого аргумента было передано \verb|-1|, а значит \verb|waitpid| сработает на любой дочерний процесс. Если в качестве второго параметра передано 0, то \verb|waitpid| будет блокирующим вызовом, а если \verb|WNOHANG| --- то нет.

Можно также перехватывать и другие сигналы, а не только \verb|$SIG{CHLD}|, напримере \verb|$SIG{INT}| (сигнал прерывания) или \verb|$SIG{ALRM}|. В частости, чтобы программа не реагировала на сочетание клавиш \verb|ctrl+C|, нужно написать:
\begin{minted}{perl}
$SIG{INT} = 'IGNORE';
\end{minted}
В этом случае программа никак не будет реагировать на это сочетание клавиш. Чтобы вывести сообщение можно поставить не \verb|'IGNORE'|, а передать ссылку на функцию, которая выводит сообщение пользователю (например, что программу не стоит резко завершать из-за возможности потери данных). Вернуть стандартное поведение можно так (например, если никакие опасные операции не выполняются):
\begin{minted}{perl}
$SIG{INT} = 'DEFAULT';
\end{minted}
После этого программа по сочетанию клавиш \verb|ctrl+C| просто завершится.

В любом месте программы можно завести будильник (один на всю программу). Если этот будильник сработал, от операционной системы приходит сигнал \verb|$SIG{ALRM}|. Например:
\begin{minted}{perl}
use Fcntl ':flock';
$SIG{ALRM} = sub {die "Timeout"};

alarm(10);

eval {
    flock(FH, LOCK_EX) or die "can't flock: $!";
};

alarm(0);
\end{minted}
Здесь программа заводит будильник на 10 секунд, после чего пытается получить lock на файл (это блокирующая операция). Если она получает lock, будильник выключается. В ином случае, после 10 секунд ожидания, программа завершается.

\subsection{Дополнительные модули}
Дополнительные модули для обеспечения взаимодействия процессов (подробнее см. документацию на CPAN):
\begin{itemize}
  \item Модули \verb|IPC::Open3| и \verb|IPC::Run3| позволяют вызвать программу (блокирующая операция) и перехватить все три файловых манипулятора:
  \begin{minted}{perl}
  my($wtr, $rdr, $err);
  $pid = open3($wtr, $rdr, $err, 'cmd', 'arg', ...);
  \end{minted}
  В результате будет порожден новый процесс, в котором будет исполнена требуемая команда с заданными аргументами. На стандартный вход будет подано содержимое \verb|$wtr|, а стандартный вывод и диагностические сообщения, полученные в результате выполнения команды, будут записаны в \verb|$rdr| и \verb|$err|.
  \item IO::Handle
\end{itemize}

Также бывает удобным создать именованный канал:
\begin{minted}{perl}
% mkfifo /path/named.pipe
\end{minted}
Этот именованный канал будет фактически файлом в файловой иерархии и может быть найден с помощью любого файлового менеджера. Получить доступ к нему из perl можно в любой момент после создания:
\begin{minted}{perl}
open( my $fifo, '<', '/path/named.pipe' );
while(<$fifo>){
    print "Got: $_";
}
close($fifo);
\end{minted}


\section{Работа с сокетами}
\subsection{Socket}
Работа с сокетами реализована посредством модуля Socket, который является простой привязкой к аналогичной библиотеке из языка Си (с точностью до имен функций):
\begin{minted}{perl}
use Socket; # include <socket.h>
\end{minted}
Следующим образом можно определить IP-адрес по известному url-адресу:
\begin{minted}{perl}
my $name = 'search.cpan.org';
my $addr_bin = gethostbyname($name);
my $ip = inet_ntoa($addr_bin);   # 194.106.223.155
\end{minted}
И в обратную сторону:
\begin{minted}{perl}
my $ip = '207.171.7.72';
my $addr_bin = inet_aton($ip);
my $name = gethostbyaddr($addr_bin, PF_INET);
                                 # mt.perl.org
\end{minted}
Для того, чтобы работать с сокетами на более высоком уровне, требуется использовать модуль \verb|IO::Socket|. Он предоставляет более <<перловый>> интерфейс для работы с сокетами, а также автоматически делает типовые действия.

Простейшее приложение-клиент, которое запрашивает страницу с сервера, имеет вид:
\begin{minted}{perl}
use strict;
use IO::Socket;
my $socket = IO::Socket::INET->new(
    PeerAddr => 'search.cpan.org',
    PeerPort => 80,
    Proto    => "tcp",
    Type     => SOCK_STREAM)
or die "Can`t connect to search.cpan.org $/";

print $socket
   "GET / HTTP/1.0\nHost: search.cpan.org\n\n";
my @answer = <$socket>;
print(join($/, @answer));
\end{minted}
Сокеты, в отличие от pipe, двунаправленные, то есть в сокет можно как писать, так и читать из него. Поэтому иногда для общения между процессами также имеет смысл использовать сокеты (например, по сети). В Unix-подобных ОС даже существуют unix-сокеты, представляющие собой именованные файлы.

Код программы-сервера несколько более громоздкий, так как необходимо предоставить множество параметров (подробнее см. в документации). Для примера приведен сервер-пингер:
\begin{minted}{perl}
use strict;
use IO::Socket;
my $server = IO::Socket::INET->new(
    LocalPort => 8081,
    Type      => SOCK_STREAM,
    ReuseAddr => 1,
    Listen    => 10)
or die "Can't create server on port 8081 : $@ $/";
while(my $client = $server->accept()){
    $client->autoflush(1);
    my $message = <$client>;
    chomp( $message );
    print $client "Echo: ".$message;
    close( $client );
    last if $message eq 'END';
}
close( $server );
\end{minted}
Команда \verb|accept| является блокирующей, ожидает подключения клиента, после чего возвращает соответствующий ему файловый манипулятор. Команда \verb|autoflush(1);| выключает любую буферизацию, что необходимо для корректной работы: при включенной буферизации небольшие сообщения от клиента будут <<висеть>> в буфере неопределенное время, пока буфер не будет заполнен. Команду \verb|autoflush(1);| не следует использовать только, если передача происходит большими фрагментами. Операция чтения сообщения от клиента является блокирующей до получения сообщения.


\subsection{Обработка нескольких соединений}
Основным недостатком предыдущего сервера является то, что к нему может подключиться максимум один клиент (после того, как accept был принят, сервер больше не ожидает подключений). Решить эту проблему можно с помощью команды \verb|fork|:
\begin{minted}{perl}
while(my $client = $server->accept()){
  my $child = fork();
  if($child){
    close ($client); next;
  }
  if(defined $child){
    close($server);
    my $other = getpeername($client);
    my ($err, $host, $service)=getnameinfo($other);
    print "Client $host:$service $/";
    $client->autoflush(1);
    my $message = <$client>;
    chomp( $message );
    print $client "Echo: ".$message;
    close( $client );
    exit;
  } else { die "Can't fork: $!"; }
}
\end{minted}
После принятия подключения от клиента, создается дочерний процесс, который общается с клиентом. Родительский процесс закрывает сокет для общения с клиентом и после этого ожидает новое подключение. Такая схема работы сервера, когда основной процесс только принимает подключения от новых клиентов и создает дочерние процессы для обслуживания их, лежит в основе большинства современных серверов.

Однако, если одновременно попросят подключения очень большое количество клиентов, будет создано соответствующее количество дочерних процессов. Количество дочерних процессоров ограничено и система может не позволить это сделать. Поэтому требуется сдедить за тем, чтобы количество активных подключений не превосходило количество подключений, которые можно обслужить в данный момент. Если количество активных подключений равно максимально допустимому, то пока их количество не уменьшится, новые подключения не должны приниматься.


\subsection{Определение имени и порта клиента}
При подключении нового клиента к серверу, строго говоря, неизвестно откуда пришел клиент, но в любом случае ему будет выданы данные. Если сервис не является общедоступным (например, внутрикорпоративный сервис, который выдает личные данные), то необходимо ограничить к нему доступ.

Определить имя и порт клиента можно с помощью следующего кода:
\begin{minted}{perl}
use IO::Socket qw/getnameinfo/;

my $other = getpeername($client);
my ($err, $host, $service) = getnameinfo($other);
print "New connection! from $host:$service $/";
\end{minted}
Функция \verb|getpeername| возвращает структуру:
\begin{minted}{c}
struct sockaddr_in {
    short          sin_family;
    unsigned short sin_port;
    struct in_addr sin_addr;
    char           sin_zero[8];
};
\end{minted}
В этой структуре содержится много информации. Для того, чтобы ограничить доступ к сервису корпоративной сетью, сначала нужно проверить, какой IP адрес отдается и, более того, чтобы исключить сценарий, когда подключение идет со взломанного компьютера внутри сети, необходимо этот IP адрес обратно превратить в имя хоста и принимать подключение, если только имя хоста, которое пришло, совпадает с полученным по IP адресу,

\section{Сериализация} % 1:20
\subsection{Понятие сериализации} % 1:20
Сериализация --- превращение объекта в поток байтов, по которому может быть восстановлен исходный объект. Этот протокол (способ сериализации и десериализации) должен быть определен вне зависимости от того, где именно будет храниться получившийся поток байт.

\begin{figure}[H] \centering
  \begin{tikzpicture}[      align=center,   minimum width  = 2cm,
      object/.style={draw,  fill=yellow!10, minimum height = 1.0cm },
      stream/.style={draw,  fill=yellow!5,  minimum height = 0.6cm },
      memory/.style={draw,  fill=blue!15,   minimum height = 1.6cm },
          db/.style={draw,  fill=red!15,    minimum height = 1.0cm },  ]

    \begin{scope}[node distance=1.5cm]
      \node[object                   ] (in)    {OBJECT};
      \node[stream, right = of in    ] (in-b)  {\small STREAM\\ OF BYTES};
      \node[db,     right = of in-b  ] (DB)    {DB};
      \node[stream, right = of DB    ] (out-b) {\small STREAM\\ OF BYTES};
      \node[object, right = of out-b ] (out)   {OBJECT};
    \end{scope}

    \begin{scope}[node distance=0.6cm]
      \node[memory, above = of DB  ] (file) {FILE};
      \node[memory, below = of DB  ] (mem)  {MEMORY};
    \end{scope}

    \begin{scope}[line width=2.2pt,shorten >=2pt, shorten <=2pt,>=stealth ]
        \draw[->] (in)    -- (in-b) node[above=1.5cm, midway] {
          \huge Serialization   };
        \draw[->] (in-b)  edge (DB) edge (file) edge (mem);
        \draw[<-] (out-b) edge (DB) edge (file) edge (mem);
        \draw[->] (out-b) --  (out) node[above=1.5cm, midway] {
          \huge Deserialization };

    \end{scope}

  \end{tikzpicture}
\end{figure}

Сериализованный объект можно передать также в программу, написанную на другом языке программирования, где он также может быть десериализован.

\subsection{Сериализация pack} % 1:22
Сериализация посредством превращения в бинарные данные по шаблону подходит для использования в высоконагруженных приложениях. Например, значения элементов массива могут быть сериализованы в строку, где между любыми двумя значениями будет расположен разделитель. Такой способ сериализации позволяет экономить память, так как любой бинарный поток, как правило, не вносит лишних символов.

Если передавать те же данные в текстовом виде (например, используя data dumper), строка будет содержать также кавычки, запятые и так далее: такой протокол очень сложно расшифровать и он не экономит ни байты при передаче, ни процессорное время.

Для превращения перловых структур в байтовые строки и обратно существуют операции \verb|pack| и \verb|unpack| соответственно:
\begin{minted}{perl}
pack TEMPLATE, LIST
\end{minted}
Функция \verb|pack| в качестве первого параметра принимает шаблон, в который надо сжимать, а в качестве второго --- список элементов, который к этому шаблону нужно применить. Шаблон представляет последовательность из символов:
\begin{verbatim}
  a  ---   строка байт, дополняемая нулями
  A  ---   строка байт, дополняемая пробелами
  b  ---   Битовая строка (младший бит идет первым)
  с  ---   Однобайтовый символ со знаком
  d  ---   Значение с плавающей запятой, двойной точности
  f  ---   Значение с плавающей запятой, одинарной точности шаблона
  h  ---   Строка шестнадцатиричных значений (младшие разряды идут    первыми)
  i  ---   Целое со знаком
  l  ---   Целое со знаком типа long
  n  ---   Целое 16 бит big-endian
  v  ---   Целое 16 бит little-endian
\end{verbatim}
При написании шаблона нужно четко придерживаться протокола, о котором есть договоренности с принимающей стороной. Существуют специальный символ \verb|/|, а также символ \verb|*|, которые позволяют записать длину строки перед самой строкой:
\begin{minted}{perl}
pack "Ca* ", length("Test"), "Test";# "\04Test"

pack "C/a*", "Test";          # "\04Test"
pack "L/a*", "Test";          # "\04\00\00\00Test"
pack "w/a*", "Test";          # "\04Test"
\end{minted}
Если на принимающей стороне программа написана на более низкоуровневом языке (например, на Си), то считав длину строки может быть сразу выделен буфер нужного размера.

Чтобы освоиться с pack, нужна небольшая тренировка. Несколько примеров сериализации и десериализации:
\begin{minted}{perl}
pack "A5", "perl", "language";      # "perl "
pack "A5 A2 A3", "perl", "language";# "perl la   "

pack "H2", "31";                    # "1"
pack "B8", "00110001"               # "1"

pack "LLxLLx", 1, 2, 3, 4;
    # "\1\0\0\0 \2\0\0\0 \0 \3\0\0\0 \4\0\0\0 \0"

unpack "H*", pack "A*", "string";   # 737472696e67

unpack "(H2)*", pack "A*", "string";
    # (73,74,72,69,6e,67)
\end{minted}

\subsection{Сериализация JSON}  % 1:30
JSON --- компактный человекочитаемый протокол обмена данными, произошедший от JavaScript. Поскольку JavaScript очень быстро сериализует-десериализует JSON, для обмена данными между браузером и web-сервером обычно используется именно JSON.

Пример данных в формате JSON:
\begin{minted}{json}
{ "orderID": 12345,
  "shopperName": "Ваня Иванов",
  "shopperEmail": "ivanov@example.com",
  "contents": [
    {
      "productID": 34,
      "productName": "Супер товар",
      "quantity": 1
    },
    {
      "productID": 56,
      "productName": "Чудо товар",
      "quantity": 3
    }
  ],
  "orderCompleted": true
}
\end{minted}
Этот протокол обмена данных очень хорошо читается человеком, и поэтому очень полезен при отладке.

Кодировать в JSON и декодировать из JSON можно с помощью \verb|encode_json| и \verb|decode_json|. Пример декодирования JSON в перловую структуру:
\begin{minted}{perl}
use JSON::XS;
use DDP;
p JSON::XS::decode_json(
    '{"key_array":["val1", "val2", 3]}'
);
\end{minted}
Перловая структура будет иметь вид:
\begin{minted}{bash}
{"key_array":["val1","val2",3]}
\ {
    key_array   [
        [0] "val1",
        [1] "val2",
        [2] 3
    ]
}
\end{minted}
Так выглядит преобразование из перловой структуры обратно в JSON:
\begin{minted}{perl}
use strict;
use JSON::XS;
my $struct = {key1 => 3};
print "Value: ".$struct->{key1}.$/;        #  Value: 3
print JSON::XS::encode_json( $struct ).$/; #  {"key1":"3"}
\end{minted}


В JSON выделяют следующие структуры:
\begin{itemize}
  \item \textbf{Объект} (как хэшу в Perl)~--- неупорядоченное множество ключей и соответствующих им значений.
  \begin{figure}[H] \centering
\begin{tikzpicture}[grammar style,setSyntaxDiagramPoints]
  \draw (LBracket)--(LP)|-(UP)-|(RP)--(RBracket);
  \draw (L)|-(DW)node[operator] {,}-|(R)--cycle;
  \draw[|-|] (LEnd)--(LBracket) node[operator] {\{}     --
      			             (-2,0) node[value]    {string} --
                         (0,0)  node[operator] {:}      --
	                       (2,0)  node[value]    {value}  --
                     (RBracket) node[operator] {\}}     -- (REnd);
\end{tikzpicture}
\end{figure}
  \item \textbf{Массив} (список в Perl)~--- упорядоченное множество значений.
\begin{figure}[H] \centering
  \begin{tikzpicture}[grammar style,setSyntaxDiagramPoints]
   \draw (LBracket)--(LP)|-(UP)-|(RP)--(RBracket);
   \draw (L)|-(DW)node[operator] {,}-|(R)--cycle;
   \draw[|-|]
      (LEnd)--(LBracket) node[operator] {[}     --
                  (0,0)  node[value]    {value}  --
              (RBracket) node[operator] {]}     -- (REnd);
  \end{tikzpicture}
\end{figure}
  \item \textbf{Значение}~--- число, строка, массив, объект или же неспецифичные для perl переменные (true, false и null). Модуль \verb|JSON::XS| обычно значение \verb|undef| в JSON отображает как \verb|null|. Работа с \verb|true| и \verb|false| идет с помощью двух соответствующих констант, которые определены внутри \verb|JSON::XS|.
\begin{figure}[H] \centering
  \begin{tikzpicture}[grammar style,setSyntaxDiagramPoints]
  	\draw[|-|] (LEnd) --  (0,0) node[value] {string} --  (REnd);

    \foreach [count=\n] \label in {number,object,array}
   	  \draw (LEnd)--(LP)|-(0,-0.8*\n) node[value] {\label}-|(RP)--(REnd);

    \foreach [count=\n from 4] \label in {true,false,null}
   	  \draw (LEnd)--(LP)|-(0,-0.8*\n) node[exotic value]{\label}-|(RP)--(REnd);
  \end{tikzpicture}
\end{figure}
\end{itemize}

\subsection{Сериализация CBOR}  % 1:35
CBOR исторически связан с JSON, но в отличие от последнего, является бинарным форматом. За счет этого скорость работы с этим форматом может быть немного выше, чем при работе с JSON:
\begin{minted}{perl}
use CBOR::XS;
my $cbor = CBOR::XS::encode_cbor([12,20,30]);
my $hash = CBOR::XS::decode_cbor( $cbor );
\end{minted}
В CBOR поддерживается потоковая передача. Это значит, что строка может состоять из нескольких записанных друг за другом валидных данных в формате CBOR и все они будут успешно распакованы.
\begin{minted}{perl}
use CBOR::XS;
my $cbors = CBOR::XS::encode_cbor([12,20,30]);
$cbors   .= CBOR::XS::encode_cbor(
    ["val1","val2","val3"]
);
my @array = ();
my $cbor_obj = CBOR::XS->new();
while( length $cbors ){
  my($data, $len)=$cbor_obj->decode_prefix($cbors);
  substr $cbors, 0, $len, '';
  push @array, $data;
}
\end{minted}
Также в формате CBOR не тратится память на кавычки, скобки и так далее, а структура данных организуется с помощью спец-символов.

\subsection{Сериализация MSGPACK}
MSGPACK --- бинарный вид сериализации, в котором целые числа переводятся в двоичное представление (а следовательно небольшие числа занимают по байту).
\begin{minted}{perl}
use strict;
use Data::MessagePack;
my $mp = Data::MessagePack->new();
my $packed   = $mp->pack({a => 1, b => 2, c => 3});
my $hash = $mp->unpack($packed);
\end{minted}
\begin{table}[H]{ \renewcommand{\arraystretch}{1.4} \centering
  \begin{tabular}{|p{3cm}|p{3cm}|p{3cm}|}\hline
  \cellcolor{red!3}&    \cellcolor{red!3}
                        \textbf{JSON}     &   \cellcolor{red!3}
                                              \textbf{MessagePack}  \\ \hline
  \cellcolor{red!3}
  \textbf{null}    & \verb|null|          &\verb|c0|                \\ \hline
  \cellcolor{red!3}
  \textbf{Integer} & \verb|10|            &\verb|0a|                \\ \hline
  \cellcolor{red!3}
  \textbf{Array}   & \verb|[20]|          &\verb|91 14|             \\ \hline
  \cellcolor{red!3}
  \textbf{String}  & \verb|"30"|          &\verb|a2 '3' '0'|        \\ \hline
  \cellcolor{red!3}
  \textbf{Map}     & \verb|"{"40":null}"| &\verb|81 a1 '4' '0' c0|  \\ \hline
\end{tabular} }
\end{table}
\textbf{MessagePack} является достаточно компактным методом сериализации.


\subsection{Сериализация Storable}
Сериализатор Storable умеет сериализовать и десериализовать какие-то структуры непосредственно в файл. В этом файле появляются бинарные данные, который на данный момент в основном понятны только perl'у:
\begin{minted}{perl}
use Storable;
my %table = ( "key1" => "val" );
store \%table, 'file';
$hashref = retrieve('file');
\end{minted}
При необходимости можно восстановить структуру по сохраненным в файл данным:
\begin{minted}{perl}
use Storable qw/freeze thaw/;
my %table = ( "key1" => "val" );
my $serialized = freeze \%table;
my $hash = thaw( $serialized );
\end{minted}
Методы freeze и thaw позволяют сериализовать и десериализовать не в файл, а в скалярную переменную.

Особо нужно отметить, что Storable очень полезен при реализации работы с файлами-конфигами.

\subsection{Сериализация XML}
Один из самых распространенных форматов для обмена данными --- это формат XML, который отличается своей хорошей читаемостью. Размер XML файлов всегда очень большой (на больших структурах занимает на порядок больше места, чем JSON).
\begin{minted}{perl}
use XML::LibXML;
my $dom = XML::LibXML->load_xml(
    string => '<xml><test>1</test></xml>'
);
\end{minted}

Замечательной особенностью парсера XML является то, что они работают поточно.
% TODO В лекции сказано, что HTML --- подмножество XML, что не совсем так. (XHTML является валидным XML)
В этом случае парсер обрабатывает строку по чуть-чуть и, если встречает тег или данные, вызывает указанную следующим образом функцию:
\begin{minted}{perl}
use XML::Parser;
my $parser = XML::Parser->new(
    Handlers => {
        Start => sub{print "New tag"},
        End   => sub{print "End tag"},
        Char  => sub{print "Data"}
    });
$parser->parse('<xml><test>1</test></xml>');
\end{minted}

\subsection{Сравнение быстродействия}
Здесь приведено быстродействие (скорость раскодирования одной и той же структуры) для всех представленных выше форматов:
\begin{minted}{bash}
YAML           84/s
XML::Simple   800/s
Data::Dumper 2143/s
FreezeThaw   2635/s
YAML::Syck   4307/s
JSON::Syck   4654/s
Storable     9774/s
JSON::XS    41473/s
CBOR::XS    42369/s
\end{minted}
Самым быстрым является CBOR, формат, пришедший на смену JSON. JSON работает только лишь чуть-чуть медленнее. Storable проигрывает на порядок, так как он заточен на работу с файловой системой.

\subsection{Сериализация Storable}
Сериализатор Storable умеет сериализовать и десериализовать какие-то структуры непосредственно в файл. В этом файле появляются бинарные данные, который на данный момент в основном понятны только perl'у:
\begin{minted}{perl}
use Storable;
my %table = ( "key1" => "val" );
store \%table, 'file';
$hashref = retrieve('file');
\end{minted}
При необходимости можно восстановить структуру по сохраненным в файл данным:
\begin{minted}{perl}
use Storable qw/freeze thaw/;
my %table = ( "key1" => "val" );
my $serialized = freeze \%table;
my $hash = thaw( $serialized );
\end{minted}
Методы freeze и thaw позволяют сериализовать и десериализовать не в файл, а в скалярную переменную.

Особо нужно отметить, что Storable очень полезен при реализации работы с файлами-конфигами.

\subsection{Сериализация XML}
Один из самых распространенных форматов для обмена данными --- это формат XML, который отличается своей хорошей читаемостью. Размер XML файлов всегда очень большой (на больших структурах занимает на порядок больше места, чем JSON).
\begin{minted}{perl}
use XML::LibXML;
my $dom = XML::LibXML->load_xml(
    string => '<xml><test>1</test></xml>'
);
\end{minted}

Замечательной особенностью парсера XML является то, что они работают поточно.
% TODO В лекции сказано, что HTML --- подмножество XML, что не совсем так. (XHTML является валидным XML)
В этом случае парсер обрабатывает строку по чуть-чуть и, если встречает тег или данные, вызывает указанную следующим образом функцию:
\begin{minted}{perl}
use XML::Parser;
my $parser = XML::Parser->new(
    Handlers => {
        Start => sub{print "New tag"},
        End   => sub{print "End tag"},
        Char  => sub{print "Data"}
    });
$parser->parse('<xml><test>1</test></xml>');
\end{minted}

\subsection{Сравнение быстродействия}
Здесь приведено быстродействие (скорость раскодирования одной и той же структуры) для всех представленных выше форматов:
\begin{minted}{bash}
YAML           84/s
XML::Simple   800/s
Data::Dumper 2143/s
FreezeThaw   2635/s
YAML::Syck   4307/s
JSON::Syck   4654/s
Storable     9774/s
JSON::XS    41473/s
CBOR::XS    42369/s
\end{minted}
Самым быстрым является CBOR, формат, пришедший на смену JSON. JSON работает только лишь чуть-чуть медленнее. Storable проигрывает на порядок, так как он заточен на работу с файловой системой.

\section{Разбор входных параметров}
У каждой запускаемой программы могут быть флаги (не нужно указывать значение):
\begin{minted}{bash}
rm -rf
ls -l
\end{minted}
и параметры (значение указывать нужно, но может быть задано значение по умолчанию):
\begin{minted}{bash}
mkdir -m 755
perl -e ''
\end{minted}

Модуль Getopt::Long позволяет удобно работать с флагами и параметрами.
\begin{minted}{perl}
use Getopt::Long;

my $param;
GetOptions("example" => \$param);
\end{minted}

\begin{table}[H]{ \renewcommand{\arraystretch}{1.4} \centering
  \begin{tabular}{|p{7cm}|p{5cm}|}\hline
 {\bf Описание параметра/флага}   & {\bf Пример}             \\ \hline
    param                         & --param или отсутствует  \\ \hline
    param!                        & --param --noparam        \\ \hline
    param=s                       & --param=string           \\ \hline
    param:s                       & --param --param=string   \\ \hline
    param=i                       & --param=1                \\ \hline
    param:i                       & --param --param=1        \\ \hline
    param=f                       & --param=3.14             \\ \hline
    param:f                       & --param --param=3.14     \\ \hline
    param                         &  p=s                     \\ \hline
\end{tabular} }
\end{table}
Его использование выглядит примерно так:
\begin{minted}{perl}
use Getopt::Long;
use Pod::Usage;
my $param = {};
GetOptions($param, 'help|?', 'man', 'verbose')
	or pod2usage(2);
pod2usage(1) if $param->{help};
pod2usage(-exitval => 0, -verbose => 2)
	if $param->{man};

__END__

=head1 NAME

sample - Script with Getopt::Long

=head1 SYNOPSIS
sample [options] [file ...]
Options:
-help            brief help message
-verbose         verbosity mode

=head1 OPTIONS

=over 8

=item B<-help>

Print help message.

=item B<-verbose>

Verbosely processing

=back

=head1 DESCRIPTION

B<this program>
Programm do something

=cut
\end{minted}
Если в хэше \verb|$param| появился ключ \verb|help|, значит при запуске приложения был передан соответствующий флаг и так далее. В этом примере использована удобная Pod-разметка для сохранения документации прямо внутри файла модуля после \verb|__END__|. Для этого используется модуль Pod::Usage, который экспортирует метод pod2usage.

\section{Интерактивный режим}
Интерактивный режим --- режим работы консольного приложения, в котором приложение не отсоединено от терминала, а значит с помощью стандартного ввода-вывода можно спросить пользователя при необходимости. Проверить, активен ли интерактивный режим можно так:
\begin{minted}{perl}
use strict;

sub is_interactive {
    return -t STDIN && -t STDOUT;
}

my $do = 1;
while( is_interactive() && $do ){
    print "Tell me anything: ";
    my $line = <>;
    print "Echo: ".$line;
    $do = 0 if $line eq "bye$/";
}

print "Goodbye$/";
\end{minted}
Если приложение не может принять решение самостоятельно, а интерактивный режим уже недоступен, оно должно упасть, так как не сможет спросить пользователя о его предпочтениях.

%\include{lectures/L6/L6}
%\include{lectures/L7/L7}
%\include{lectures/L8/L8}
%\include{lectures/L9/L9}
%\include{lectures/L10/L10}
%\include{lectures/L11/L11}
%\include{lectures/L12/L12}
\end{document}
